\documentclass[12pt, titlepage]{article}
\usepackage[german]{babel}
\usepackage[utf8]{inputenc}
\usepackage[T1]{fontenc}
\usepackage{color}
\usepackage{amssymb}
\usepackage{amsthm}
\usepackage{graphicx}
\usepackage{chngcntr}
\usepackage{ upgreek }
\usepackage{mathtools}
\usepackage{empheq}
\usepackage{amsmath,amssymb,amsthm,mathtools}
\usepackage{listings} 
\usepackage{braket}
\usepackage{tikz}
\usetikzlibrary{arrows,shapes,calc}
\lstset{numbers=left, numberstyle=\tiny, numbersep=5pt} \lstset{language=Scilab} 

%\\textcolor{red}{}
%Formel-Farbe
%\textcolor[RGB]{0,0,204}{\Phi sik}
%Präambel für römische Zahlen 
\newcommand{\RM}[1]{\MakeUppercase{\romannumeral #1}}


\begin{document}

%\\\textcolor{red}{Hier fehlt was}\\
%\maketitle %gibt dastitelblatt hier aus
\begin{equation}
\textcolor[RGB]{0,0,204}{\Phi sik}
\end{equation}

\section{Tests}
\begin{equation*}
   x \underset{\mathrm{def}}{=} y
\end{equation*}

\begin{equation*}
   x
   \tikz[baseline=-1pt]{
     \node (eq)
     {$=$};
   }
   y
\end{equation*}


\begin{tikzpicture}[overlay]
   \node (t) at ($(eq) + (-0.8,-0.5)$) {\footnotesize Explanation};
   \path[->,shorten >= 5pt] (eq.base) edge [bend left=10] (t.mid) ;
\end{tikzpicture}

\begin{equation*}
  x
  % with tikzpicture environment
  \underset{\mathllap{
    \begin{tikzpicture}
      \draw[->] (-0.3, 0) to[bend right=20] ++(0.3,2ex);
      \node[below left] at (0,0) {because $x = y$};
    \end{tikzpicture}
  }}{=}
  y + x - x
  % with tikz command
  \overset{\mathclap{\tikz \node {$\downarrow$} node [above=1ex] {trivial};}}{=}
  y + y - y
\end{equation*}



\end{document}