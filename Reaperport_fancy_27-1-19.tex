%%%%%%%%%%%%%%%%%%%%%%%%%%%%%%%%%%%%%%%%%
% Masters/Doctoral Thesis 
% LaTeX Template
% Version 2.5 (27/8/17)
%
% This template was downloaded from:
% http://www.LaTeXTemplates.com
%
% Version 2.x major modifications by:
% Vel (vel@latextemplates.com)
%
% This template is based on a template by:
% Steve Gunn (http://users.ecs.soton.ac.uk/srg/softwaretools/document/templates/)
% Sunil Patel (http://www.sunilpatel.co.uk/thesis-template/)
%
% Template license:
% CC BY-NC-SA 3.0 (http://creativecommons.org/licenses/by-nc-sa/3.0/)
%
%%%%%%%%%%%%%%%%%%%%%%%%%%%%%%%%%%%%%%%%%

%----------------------------------------------------------------------------------------
%	PACKAGES AND OTHER DOCUMENT CONFIGURATIONS
%----------------------------------------------------------------------------------------

\documentclass[
11pt, % The default document font size, options: 10pt, 11pt, 12pt
%oneside, % Two side (alternating margins) for binding by default, uncomment to switch to one side
english, % ngerman for German
singlespacing, % Single line spacing, alternatives: onehalfspacing or doublespacing
%draft, % Uncomment to enable draft mode (no pictures, no links, overfull hboxes indicated)
%nolistspacing, % If the document is onehalfspacing or doublespacing, uncomment this to set spacing in lists to single
%liststotoc, % Uncomment to add the list of figures/tables/etc to the table of contents
%toctotoc, % Uncomment to add the main table of contents to the table of contents
%parskip, % Uncomment to add space between paragraphs
%nohyperref, % Uncomment to not load the hyperref package
headsepline, % Uncomment to get a line under the header
%chapterinoneline, % Uncomment to place the chapter title next to the number on one line
%consistentlayout, % Uncomment to change the layout of the declaration, abstract and acknowledgements pages to match the default layout
]{MastersDoctoralThesis} % The class file specifying the document structure

\usepackage[utf8]{inputenc} % Required for inputting international characters
\usepackage[T1]{fontenc} % Output font encoding for international characters

%\usepackage[ngerman,english]{babel}

\usepackage{color}
\usepackage{amssymb}
\usepackage{amsthm}
\usepackage{graphicx}
%\usepackage{chngcntr}
\usepackage{upgreek}
\usepackage{mathtools}
\usepackage{empheq}
\usepackage{amsmath,amssymb,amsthm,mathtools}
\usepackage{listings} 
\usepackage{braket}
\usepackage{tikz}
\usepackage[toc,page]{appendix}
\usetikzlibrary{arrows,shapes,calc}
\lstset{numbers=left, numberstyle=\tiny, numbersep=5pt} \lstset{language=Scilab} 
%%\tolerance=700 %Soll Badbox entfernen klappt nicht
%\\textcolor{red}{}
%Formel-Farbe
%\textcolor[RGB]{0,0,204}{\Phi sik}
%Präambel für römische Zahlen 
\newcommand{\RM}[1]{\MakeUppercase{\romannumeral #1}}
%Command for tighter \ldots
\newcommand\mydots{\makebox[1em][c]{.\hfil.\hfil.}}

\usepackage{mathpazo} % Use the Palatino font by default

%\usepackage[backend=bibtex,style=authoryear,natbib=true]{biblatex} % Use the bibtex backend with the authoryear citation style (which resembles APA)

%\addbibresource{example.bib} % The filename of the bibliography

%\usepackage[autostyle=true]{csquotes} % Required to generate language-dependent quotes in the bibliography

%----------------------------------------------------------------------------------------
%	MARGIN SETTINGS
%----------------------------------------------------------------------------------------

\geometry{
	paper=a4paper, % Change to letterpaper for US letter
	inner=2.5cm, % Inner margin
	outer=3.8cm, % Outer margin
	bindingoffset=.5cm, % Binding offset
	top=1.5cm, % Top margin
	bottom=1.5cm, % Bottom margin
	%showframe, % Uncomment to show how the type block is set on the page
}

%----------------------------------------------------------------------------------------
%	THESIS INFORMATION
%----------------------------------------------------------------------------------------

\thesistitle{Specialization report} % Your thesis title, this is used in the title and abstract, print it elsewhere with \ttitle
\supervisor{Dr. James \textsc{Smith}} % Your supervisor's name, this is used in the title page, print it elsewhere with \supname
\examiner{} % Your examiner's name, this is not currently used anywhere in the template, print it elsewhere with \examname
\degree{Doctor of Philosophy} % Your degree name, this is used in the title page and abstract, print it elsewhere with \degreename
\author{John \textsc{Smith}} % Your name, this is used in the title page and abstract, print it elsewhere with \authorname
\addresses{} % Your address, this is not currently used anywhere in the template, print it elsewhere with \addressname

\subject{Biological Sciences} % Your subject area, this is not currently used anywhere in the template, print it elsewhere with \subjectname
\keywords{} % Keywords for your thesis, this is not currently used anywhere in the template, print it elsewhere with \keywordnames
\university{\href{http://www.tp1.hhu.de/}{ Heinrich Heine Universit\"at D\"usseldorf\\ Institut f\"ur Theoretische Physik I}} % Your university's name and URL, this is used in the title page and abstract, print it elsewhere with \univname
\department{\href{http://department.university.com}{Department or School Name}} % Your department's name and URL, this is used in the title page and abstract, print it elsewhere with \deptname
\group{\href{http://researchgroup.university.com}{Research Group Name}} % Your research group's name and URL, this is used in the title page, print it elsewhere with \groupname
\faculty{\href{http://faculty.university.com}{Faculty Name}} % Your faculty's name and URL, this is used in the title page and abstract, print it elsewhere with \facname

\AtBeginDocument{
\hypersetup{pdftitle=\ttitle} % Set the PDF's title to your title
\hypersetup{pdfauthor=\authorname} % Set the PDF's author to your name
\hypersetup{pdfkeywords=\keywordnames} % Set the PDF's keywords to your keywords
}

\begin{document}

\frontmatter % Use roman page numbering style (i, ii, iii, iv...) for the pre-content pages

\pagestyle{plain} % Default to the plain heading style until the thesis style is called for the body content

%----------------------------------------------------------------------------------------
%	TITLE PAGE
%----------------------------------------------------------------------------------------

\begin{titlepage}
\begin{center}

\vspace*{.06\textheight}
{\scshape\LARGE \univname\par}\vspace{1.5cm} % University name
%\textsc{\Large Doctoral Thesis}\\[0.5cm] % Thesis type

\HRule \\[0.4cm] % Horizontal line
{\huge \bfseries \ttitle\par}\vspace{0.4cm} % Thesis title
\HRule \\[1.5cm] % Horizontal line
 
{\Large \textit{Marius Thei\ss{}en \\ Matrn.: 2163903}}\\[0.5cm] 
% 
%\begin{minipage}[t]{0.4\textwidth}
%\begin{flushleft} \large
%\emph{Author:}\\
%\href{http://www.johnsmith.com}{\authorname} % Author name - remove the \href bracket to remove the link
%\end{flushleft}
%\end{minipage}
%\begin{minipage}[t]{0.4\textwidth}
%\begin{flushright} \large
%\emph{Supervisor:} \\
%\href{http://www.jamessmith.com}{\supname} % Supervisor name - remove the \href bracket to remove the link  
%\end{flushright}
%\end{minipage}\\[3cm]
% 
\vfill
\includegraphics[scale=0.11]{Logo} % University/department logo - uncomment to place it

%\large \textit{A thesis submitted in fulfillment of the requirements\\ for the degree of \degreename}\\[0.3cm] % University requirement text
%\textit{in the}\\[0.4cm]
%\groupname\\\deptname\\[2cm] % Research group name and department name
% 
\vfill

{\large \today}\\[4cm] % Date
 
\vfill
\end{center}
\end{titlepage}

%\cleardoublepage

\mainmatter % Begin numeric (1,2,3...) page numbering

\pagestyle{thesis} % Return the page headers back to the "thesis" style

% Include the chapters of the thesis as separate files from the Chapters folder
% Uncomment the lines as you write the chapters
\tableofcontents
\newpage
%\include{Chapters/Chapter1}
%\include{Chapters/Chapter2} 
%\include{Chapters/Chapter3}
%\include{Chapters/Chapter4} 
%\include{Chapters/Chapter5} 

%----------------------------------------------------------------------------------------
%	THESIS CONTENT - APPENDICES
%----------------------------------------------------------------------------------------
\chapter{Introduction}\label{Introduction}
%This report has three main goals. First we shall establish the transition between pictures in quantum mechanics and quantum field theory. We go over the more common ones, Schrödinger, Heisenberg and Interaction picture, to derive and justify the two In and Out Pictures.
%By working out a few handy techniques and methods on the way, we will proof the Gell-Mann Low formula. This important formula allows us transition of polynomial terms of field operators in the Heisenberg picture to multiple different but equivalent picture expressions. These are based on boundary conditions imposing free motion on particles along them. We derive them from experimental and physical point of views. The Gell-Mann Low formula does this to this point without loss of information. In case of calculating multi particle processes the Gell-Mann Low formula allows us to perform perturbation theory on the correlation functions by introducing the scattering operator $ S $ which then can be expanded. This  $ S $ is the key to calculate cross-sections $ \sigma $ and decay rates $ \Gamma $ .
%\textcolor{red}{What is the GellMann Law formula, why is it so important explain in worth}
Quantum field theory is a formalism that allow us to describe the interaction between elementary particles. Of great interest are observable quantities like cross-sections or decay rates, which can be established theoretically and contrasted with experimental results. Both observables can be determined from the amplitude of the corresponding process, which are calculated by using pertubation theory. The main goal of this report is the derivation of the Gell-Mann Low formula that enables us  to apply this technique.
 
In chapter \ref{Pictures in Quantum Mechanics} we discuss the more common Schrödinger, Heisenberg and Interaction picture. Followed up by $ in $ and $ out $ pictures to which Gell-Mann Low formula allows easy transition. These pictures are based on boundary conditions, which are motivated in section \ref{in_out_picture_external_currents}, imposing free motion on particles along them. We derive them from experimental and physical points of view. After deriving concrete solutions for the unitary operators associated with these new pictures, we define the scattering operator $ \hat{S} $ in chapter \ref{Scattering operator} and establish the Gell-Mann Low formula.
\chapter{Pictures in Quantum Mechanics }\label{Pictures in Quantum Mechanics}
\section{Schrödinger picture and Heisenberg picture}\label{SHpicture}
The choice of a picture always requires to establish the states but also the corresponding operators. 
In the Schrödinger picture the operators are time-independent but the wavefunctions are time dependent. The time evolution of a state vector is controlled by the Schrödinger equation. Let $ \ket{\Psi(t)} $ denote a state vector at time $ t $. It satisfies
\begin{equation}\label{eq:schroedinger}
\textcolor[RGB]{0,0,204}{
i \hbar \frac{\partial }{\partial t} \ket{\Psi_{S}(t)} =
\hat{H} \ket{\Psi_{S}(t)},
}
\end{equation}
where $ \hat{H} $ is the Hamiltonian of the system. When assuming it time independent, the solution of Eq.\enskip\eqref{eq:schroedinger} can be formally written as 
\begin{equation}\label{eq:time_ev_op_schr}
\textcolor[RGB]{0,0,204}{
\ket{\Psi_{S}(t)} 
=\hat{U}(t-t_{0})\ket{\Psi_{S}(t_0)}
}
\end{equation}
with $ \hat{U}(t-t_{0}) = e^{-\frac{i}{\hbar}\hat{H}(t-t_{0})} $ the time evolution operator, which satisfies the differential equation 
\begin{equation}\label{evo_U_1}
i\hbar\partial_{t}\hat{U}(t-t_{0})=\hat{H}\hat{U}(t-t_{0})
. 
\end{equation}
Under the general assumption of the Hamiltonian being hermitian $ \hat{U}(t-t_{0}) $  is also an unitary operator, meaning:
\begin{equation}\label{Unitary_U}
\begin{split}
&\hat{U}(t-t_{0}) \times \hat{U}^{\dagger}(t-t_{0}) = e^{-\frac{i}{\hbar}\hat{H}(t-t_{0})} e^{\frac{i}{\hbar}\hat{H}(t-t_{0})}=1
\\
&=\hat{U}(t-t_{0}) \times \hat{U}^{-1}(t-t_{0})
\end{split}
\end{equation}
 Going back to Eq.\enskip\eqref{eq:time_ev_op_schr} we see $ \ket{\Psi_{S}(t_0)} $ is a ket of $ t=t_{0} $. We shall generally take $ t_{0}=0 $ and write
\begin{equation}\label{eq:phi_s_and_phi_h}
\textcolor[RGB]{0,0,204}{
\ket{\Psi_{S}(t)} 
= e^{-\frac{i}{\hbar}\hat{H}t}
\ket{\Psi_{H}}.
}
\end{equation}
The state on the right-hand side has no longer time dependence. This defines the  state in the Heisenberg picture.

The above two pictures differ between each other in the way of storing the time dependence. In the Schrödinger picture only the states carry such a dependence, whereas in the Heisenberg picture only operators has this possibility. To verify this statement we study the matrix element of an operator in the Schrödinger picture
\begin{equation}\label{eq:matrix_ele_time_dep}
\textcolor[RGB]{0,0,204}{
\bra{\Psi^{\prime}_{S}(t)}\hat{A}^{S}\ket{\Psi_{S} (t)}
=\bra{\Psi^{\prime}_{H}}e^{\frac{i}{\hbar}t\hat{H}}\hat{A}^{S}
e^{-\frac{i}{\hbar}t\hat{H}}\ket{\Psi_{H} },
}
\end{equation}
where Eq.\enskip\eqref{eq:phi_s_and_phi_h} has been used. As a consequence, 
\begin{equation}\label{eq:S-H-Operator_Trafo}
\textcolor[RGB]{0,0,204}{
\hat{A}^{H}(t)=e^{\frac{i}{\hbar}t\hat{H}}\hat{A}^{S}
e^{-\frac{i}{\hbar}t\hat{H}}=\hat{U}(t)^{-1}\hat{A}^{S}\hat{U}(t).
}
\end{equation}
This new operator $ \hat{A}^{H}(t) $ in combination with the state $ \vert\Psi_{H} \rangle$ defines the Heisenberg picture. Observe that the time evolution of $ \hat{A}^{H}(t) $  is dictated by an equation that follows from differentiating the equation above with respect to $ t $:
\begin{equation}
\frac{d}{dt}\hat{A}^{H}(t)
=\frac{i}{\hbar}\hat{H}
e^{\frac{i}{\hbar}t \hat{H}}
\hat{A}^{S}
e^{-\frac{i}{\hbar}t \hat{H}}
%+
%e^{\frac{i}{\hbar}t \hat{H}}
%\frac{\partial\hat{A}^{S}}{\partial t}
%e^{-\frac{i}{\hbar}t \hat{H}}
+e^{\frac{i}{\hbar}t \hat{H}}
\hat{A}^{S}
\left( -\frac{i}{\hbar}\hat{H}\right) 
e^{-\frac{i}{\hbar}t \hat{H}}.
%\text{Assuming the operator has no dependence in time independent of the picture, }
%\begin{align}
%\frac{d}{dt}\hat{A}^{H}(t)
%&=\frac{i}{\hbar}\hat{H}
%e^{\frac{i}{\hbar}t \hat{H}}
%\hat{A}^{S}
%e^{-\frac{i}{\hbar}t \hat{H}}
%+
%e^{\frac{i}{\hbar}t \hat{H}}
%\hat{A}^{S}
%\left( -\frac{i}{\hbar}\hat{H}\right) 
%e^{-\frac{i}{\hbar}t \hat{H}}
%&\\
%&=\frac{i}{\hbar}
%e^{\frac{i}{\hbar}t \hat{H}}
%\left( 
%\hat{H}\hat{A}^{S}- \hat{A}^{S} \hat{H}
%\right) 
%e^{-\frac{i}{\hbar}t \hat{H}}.
\end{equation}
Here we have used the  time evolution equation \eqref{evo_U_1}. Hence,
\begin{equation}\label{time_evo_seven}
\begin{split}
&\frac{d}{dt}\hat{A}^{H}(t)
	=\frac{i}{\hbar}
	\hat{U}(t)^{-1}\hat{H}\hat{A}^{S}\hat{U}(t)
	-
	\frac{i}{\hbar}
	\hat{U}(t)^{-1}\hat{A}^{S}\hat{H}\hat{U}(t)
	\\
&\qquad\qquad=\frac{i}{\hbar}
	\hat{U}(t)^{-1}\hat{H}\underbrace{\hat{U}(t)\hat{U}(t)^{-1}}_{=1}\hat{A}^{S}\hat{U}(t)
	-
	\frac{i}{\hbar}
	\hat{U}(t)^{-1}\hat{A}^{S}\underbrace{\hat{U}(t)\hat{U}(t)^{-1}}_{=1}\hat{H}\hat{U}(t).
\end{split}
\end{equation}
The inserted $ 1 $ allows us to express Eq.\enskip\eqref{time_evo_seven} in term of operators in the Heisenberg picture.
\begin{equation}
\frac{d}{dt}\hat{A}^{H}(t)
	=\frac{i}{\hbar}
	\hat{H}^{H}(t)\hat{A}^{H}(t)
	-
	\frac{i}{\hbar}
	\hat{A}^{H}(t)\hat{H}^{H}(t),
\end{equation} 
where $ \hat{H}^{H}(t) $ is the respective Hamiltonian in the Heisenberg  picture.
Therefore:
\begin{equation}\label{eq:Heisenberg_time_dep._eq.}
\textcolor[RGB]{0,0,204}{
i\frac{d}{d t}\hat{A}^{H}(t)=
\frac{1}{\hbar}\left[ \hat{A}^{H}(t),\hat{H}^{H}(t)\right] 
.
}
\end{equation} 
\section{Interaction picture}\label{Interactionpicture}
A third picture can be introduced: the Interaction picture (sometimes called the Dirac picture). We will see very shortly that, in the Interacting picture both the states and the respective operators are time dependent.
Let us suppose that the Hamiltonian in the Schrödinger picture can be splitted as follows $ \hat{H} = \hat{H}_{0}+\hat{V} $. Normally $ \hat{H}_{0} $ describe the free motion of a system, whereas  $ \hat{V} $ represents its interaction, which could be with an external source. Although it often used in a perturbative approach, the Interaction picture does not require $ \hat{V} $  to be small as compared with $ \hat{H}_{0} $. 
Inserting this decomposition of $ \hat{H} $ in the unitary operator introduced below Eq.\enskip\eqref{eq:time_ev_op_schr}:
\begin{equation}\label{eq:omega_and_U}
\textcolor[RGB]{0,0,204}{
\hat{U}(t)=e^{-\frac{i}{\hbar}t \hat{H}}
=e^{-\frac{i}{\hbar}t\left(  \hat{H}_{0}+V\right) }
=
e^{-\frac{i}{\hbar}t \hat{H_{0}}}
\hat{\Omega}_{I} (t)
}
\end{equation}
This expression helps us to establish a formula from which operators and states in the interaction picture can be defined\footnote{By substituting Eq.\enskip\eqref{eq:omega_and_U} in Eq.\enskip\eqref{Unitary_U} one can see directly that $ \Omega_{I}(t) $ is also unitary}. For this, consider a matrix element $ 	\bra{\Psi_{S}^{\prime}(t)}
	\hat{A}^{S}
	\ket{\Psi_{S}(t)} $. Taking into account Eq.\enskip\eqref{eq:phi_s_and_phi_h} and \eqref{eq:omega_and_U} we find
%\begin{align}\label{eq:matrix_ele_for_interaction}
%\textcolor[RGB]{0,0,204}{
%	\bra{\Psi^{\prime}(t)}
%	\hat{A}_{S}
%	\ket{\Psi(t)}
%	=
%	\bra{\Psi^{\prime}}
%	(e^{-\frac{i}{\hbar}t \hat{H_{0}}}	
%	\Omega_{I} (t))^{\dagger}
%	\hat{A}_{S}
%	e^{-\frac{i}{\hbar}t \hat{H_{0}}}
%	\Omega_{I} (t)
%	\ket{\Psi}
%}
%\end{align}
\begin{subequations}
\textcolor[RGB]{0,0,0}{
\begin{align}\label{eq:matrix_ele_for_interaction}
	\bra{\Psi_{S}^{\prime}(t)}
	\hat{A}^{S}
	\ket{\Psi_{S}(t)}
  		&= 	\bra{\Psi_{H}^{\prime}}
			(e^{-\frac{i}{\hbar}t \hat{H_{0}}}	
			\Omega_{I} (t))^{\dagger}
			\hat{A}^{S}
			e^{-\frac{i}{\hbar}t \hat{H_{0}}}
			\Omega_{I} (t)
			\ket{\Psi_{H}}
  		\\
  		&= \bra{\Psi^{\prime}_{H}}
  			\Omega_{I}(t)^{-1}
			\hat{A}^{I}(t)
			\Omega_{I} (t)
			\ket{\Psi_{H}}	
			.
\end{align}
}
\end{subequations}
Here the operator in the interaction picture reads
\begin{equation}\label{eq:operator_interac_schrodinger}
\textcolor[RGB]{0,0,204}{
	\hat{A}^{I}(t)
	=
	e^{+\frac{i}{\hbar}t \hat{H_{0}}}
	\hat{A}^{S}
	e^{-\frac{i}{\hbar}t \hat{H_{0}}}	
,
}
\end{equation}
whereas a corresponding state in this picture is
\begin{equation}\label{eq:state_interac_schrodinger}
\textcolor[RGB]{0,0,204}{
	\ket{\Psi_{I}(t)}
	=\Omega_{I} (t)
			\ket{\Psi_{H}}
		.
}
\end{equation}
At the level of operators, the connection between the Interaction and the Heisenberg picture is established by inverting Eq.\enskip\eqref{eq:S-H-Operator_Trafo} and inserting the resulting $ 	\hat{A}^{S}
 $ into Eq.\enskip\eqref{eq:operator_interac_schrodinger}. This leads to
\begin{subequations}
\begin{align}
	\hat{A}^{I}(t)
	&=
	e^{+\frac{i}{\hbar}t \hat{H_{0}}}
	\hat{U}(t)	
	\hat{A}^{H}(t)
	\hat{U}(t)^{-1}	
	e^{-\frac{i}{\hbar}t \hat{H_{0}}}	.
	\\
	&=
	e^{+\frac{i}{\hbar}t \hat{H_{0}}}
	e^{-\frac{i}{\hbar}t\hat{H}}
	\hat{A}^{H}(t)
	e^{\frac{i}{\hbar}t\hat{H}}	
	e^{-\frac{i}{\hbar}t \hat{H_{0}}}	
	,
\end{align}
\end{subequations}
%%
ending with
\begin{equation}\label{eq:operator_interac_heisenberg}
\textcolor[RGB]{0,0,204}{
	\hat{A}^{I}(t)
	=
	\hat{\Omega}_{I}(t)
	\hat{A}^{H}(t)
	\hat{\Omega}_{I}(t)^{-1}
	.
}
\end{equation}
%%
The time evolution equation for $ 	\hat{A}^{I}(t) $ can be found as done for $ \hat{A}^{H}(t) $ $ [ $see below Eq.\enskip\eqref{eq:S-H-Operator_Trafo}]:
\begin{equation}\label{eq:time_evo_equ_Intera}
\textcolor[RGB]{0,0,204}{
	i\hbar
	\frac{\partial}{\partial t}
	\hat{A}^{I}
	=
	\left[ 
	\hat{A}^{I},
	\hat{H}_{0}
	\right] .
}
\end{equation}
Furthermore, an equation for $ \hat{\Omega}_{I}(t) $ can be determined. To this end we invert Eq.\enskip\eqref{eq:omega_and_U} and express $  \hat{\Omega}_{I}(t) 
  		= e^{\frac{i}{\hbar}t \hat{H_{0}}}
			\hat{U}(t)  $ . Afterwards we differentiate with respect to times: 
\begin{subequations}
\begin{align}
  		i\hbar\partial_{t}\hat{\Omega}_{I}(t) 
  		 &= 
  		 e^{\frac{i}{\hbar}t\hat{H_{0}}}
  		 \left(i\hbar\partial_{t}\hat{U}(t) \right)
  		 -
  		 \hat{H_{0}}
   		 e^{\frac{i}{\hbar}t\hat{H_{0}}}
 		 \hat{U}(t)
  		 \\
  		 &=
  		 \hat{H}
  		  e^{\frac{i}{\hbar}t\hat{H_{0}}}  		 
  		 \hat{U}(t)  		 
  		 -
  		 \hat{H_{0}}
  		   e^{\frac{i}{\hbar}t\hat{H_{0}}}
  		 \hat{U}(t),
\end{align}
\end{subequations}
where Eq.\enskip\eqref{evo_U_1} has been used. Using the definition of $ \hat{\Omega}_{I}(t) $ we end up with
\begin{equation}\label{eq:time_evo_Omega_interaction}
\textcolor[RGB]{0,0,204}{
	i\hbar
	\frac{\partial}{\partial t}
	\hat{\Omega}_{I}(t)
	=
	\hat{V}_{I}(t)
	\hat{\Omega}_{I}(t)
.}
\end{equation}
%requiring $ \hat{\Omega}_{I}(0)=1 $.
%With he boundary condition a well defined solution can be found. It stems from the condition on the time evolution operator $ U(t-t_0) $. It needed to fulfill $ U(t_0-t_0)=1 $. Since we set $ t_0 =0 $ $ \Omega_{I} $ needed this form to not collide with any equations retroactively.
To find a well defined solution, an initial condition is needed. In Eq.\enskip\eqref{eq:omega_and_U} we see that at $ t=0 $, the time evolution operator reduces to $ \hat{U}(0)=1 $, and from this the following condition $ \hat{\Omega}_{I}(0) = 1 $ arises.
We remark that $ \hat{V}_{I}(t) $ in the Interaction picture as introduced above does not require $ \hat{V} $ to be of any specific form but can still be applied in presence of external sources. 
%Additionally we can derive a differential equation for $ \hat{U}(t) $
%\begin{subequations}
%\begin{align}
%	i\hbar
%	\partial_{t}
%	(e^{-\frac{i}{\hbar}tH_{0}}
%	\Omega_{I}(t))
%		&=
%		i\hbar(-\frac{i}{\hbar}H_{0}) e^{-\frac{i}{\hbar}tH^{0}} \Omega_{I}(t)
%		+
%		i\hbar e^{-\frac{i}{\hbar}tH_{0}} \partial_{t} \Omega_{I}(t)
%		\\
%			i\hbar \partial_{t} U
%			&=H_{0}U+i\hbar e^{-\frac{i}{\hbar}tH_{0}}\partial_{t}\Omega_{I}(t).
%\end{align}
%\end{subequations}
\section{The $ \pmb{in} $ and $ \pmb{out}$ picture: External currents}\label{in_out_picture_external_currents}
Consider the set-up of most experiments in elementary particle and nuclear physics. Several particles approach each other from a macroscopic scale and interact in a microscopic section comparable to the Compton wavelength of the incoming particles. On this scale vacuum fluctuations are no longer negligible for the dynamic of the involved particles and make them impossible to distinguish between each other. As a result, the products of the interaction spread up to a macroscopic distances and the distinguishability between outgoing particles is admitted. Therefore, at such asymptotically distances, the description of the incoming and outgoing multi-particle states can be approached by direct products of single-particle effectively non-interacting states.

To bring this concept into our formulation let's consider 
%$ V=j(t)\hat{q}(t) $ with $ \hat{q}(t) $ as the operator of position and $ %j $ an external source. This will resulting in an unstable vacuum. 
%In a system with an external current a pure vacuum state can evolve over time into a multi particle state .
%Starting with 
the action of a scalar field $ \Phi $ with mass $ m=m_{0}c/\hbar $ coupled to an external source $ j(\underline{x},t) $\footnote{From now on we will work in natural units and set $ c=\hbar=1 $ }:
\begin{equation}\label{action_with_scalar_ex_current}
I=\int d^{4}x \ \mathcal{L}(\Phi, \dot{\Phi},j)=
\int d^{4}x 
\left(
\frac{1}{2}\partial_{\mu}\Phi\partial^{\mu}\Phi
-\frac{1}{2}m^{2}\Phi^{2}
-\Phi j
 \right)
 .
\end{equation}
Taking the functional derivative with respect to $ \Phi $ and setting it to zero, we obtain the equation of motion
\begin{equation}\label{motion_scalar}
\textcolor[RGB]{0,0,204}{
\left(
\partial^{2}+m^{2}
 \right)\Phi
 =j
 .}
\end{equation}
%We will confine the field in a box and impose periodic boundary conditions on it. This %is called quantisation in a box. It allows us to write the field in terms of modes.
To proceed, we quantize our field in a box of volume $ V $ and length $ L $. The classical field and its canonical momentum $ \Pi = \partial \mathcal{L} /\partial\dot{\Phi}(\underline{x},t)=\dot{\Phi}(\underline{x},t) $ are then promoted to operators $ \hat{\Phi}(\underline{x},t) $ and  $ \hat{\Pi}(\underline{x},t) $ in the Heisenberg picture. Satisfying the equal-time commutator:
\begin{equation}
\left[
\hat{\Phi}^{H}(\underline{x},t),\hat{\Pi}^{H}(\underline{x}',t)
 \right] 
 =
 i
 \delta^{3}
 (\underline{x} - \underline{x}')
 .
\end{equation}
We then expand the field operator as follows:
\begin{equation}\label{box_quanta}
\textcolor[RGB]{0,0,204}{
\hat{\Phi}^{H} (\underline{x},t)= \sum_{\underline{k}} \hat{q}^{H}_{\underline{k}}(t)u_{\underline{k}}(\underline{x})
 .}
\end{equation}
The 3 dim. wave vector $ \underline{k} $ for the modes is represented by $ \underline{k} = \frac{2\pi}{L}(n_{x},n_{y},n_{z}) $ with $ n_{i}\in \mathbb{Z} $ . 
	In this separated time and space dependency, we choose the Fourier basis for $ u_{\underline{k}}(\underline{x}) $
\begin{equation}\label{fourierbasis}
u_{\underline{k}}(\underline{x})
=
\dfrac{1}{L^{3/2}} e^{i\underline{k}\cdot \underline{x}}
,
\end{equation}
where the volume $ L^{3} $ provides the required normalization. We remark that $ u_{\underline{k}}(\underline{x}) $ constitutes an orthonormalized basis in the Hilbert space
\begin{equation}\label{ortho_relation}
\int d^{3}x \ 
u^{\ast}_{\underline{k}'}(\underline{x})
u_{\underline{k}}(\underline{x})
=
\delta_{\underline{k}, \underline{k}'}
\end{equation}
\begin{equation}\label{completness_relation}
\sum_{\underline{k}} \ 
u^{\ast}_{\underline{k}}(\underline{x})
u_{\underline{k}}(\underline{x}')
=
\delta^{3}\left(\underline{x}-\underline{x}'\right)
.
\end{equation}
We now substitute Eq.\enskip\eqref{fourierbasis} into the equation of motion \eqref{motion_scalar}
%\begin{equation}
%%\left(
%%\partial^{2}+m^{2}
%% \right)
%% \sum_{\underline{k}} \hat{q}^{H}_{\underline{k}}(t)u_{\underline{k}}(\underline{x})
%% =j(\underline{x},t)
%%	&\\
% \sum_{\underline{k}}
% \left[ 
%	\left(  
%	\dfrac{\partial}{\partial t^{2}}
%	-\nabla^{2}
%	+m^{2}
%	\right)  
%	\hat{q}^{H}_{\underline{k}}(t)u_{\underline{k}}(\underline{x})
% \right] 
% =j(\underline{x},t).
%\end{equation}
%The basis will enable us to take the spacial derivative
. As a consequence  
\begin{equation}
 \sum_{\underline{k}}
 \left[ 
	\ddot{\hat{q}}^{H}_{\underline{k}}(t)u_{\underline{k}}(\underline{x})
	+\underline{k}^{2}\hat{q}^{H}_{\underline{k}}(t)u_{\underline{k}}(\underline{x})
	+m^{2}\hat{q}^{H}_{\underline{k}}(t)u_{\underline{k}}(\underline{x})
 \right] 
  =j(\underline{x},t).
\end{equation}
To get an equation for $ \hat{q}^{H}_{\underline{k}}(t) $ alone we need to get rid of $ u_{\underline{k}}(\underline{x}) $ and remove the space dependence in the current. Multiplying with $ u^{\ast}_{\underline{k}'}(\underline{x}) $ and integrating over the whole space we find,
%\begin{equation}
%\int d^{3}\underline{x} \ 
%u^{\ast}_{\underline{k}'}
%\ \cdot
%\vert 
% \sum_{\underline{k}}
% \left[ 
%	\ddot{\hat{q}}_{\underline{k}}(t)u_{\underline{k}}(\underline{x})
%	-(i)^{2}\underline{k}^{2}\hat{q}_{\underline{k}}(t)u_{\underline{k}}(\underline{x})
%	+m^{2}\hat{q}_{\underline{k}}(t)u_{\underline{k}}(\underline{x})
% \right] 
% =
% \int d^{3}\underline{x} \ 
%j(\underline{x},t) \dfrac{1}{\sqrt{V}} e^{i\underline{k}\underline{x}}
%\end{equation}
\begin{equation}
 \sum_{\underline{k}}
 \left[ 
 \int d^{3}x \ u^{\ast}_{\underline{k}'}u_{\underline{k}}
 \left( 
 \ddot{\hat{q}}^{H}_{\underline{k}}(t) 
 +
 \left( \underline{k}^{2}+m^{2}\right) 
 \hat{q}^{H}_{\underline{k}}(t) 
 \right) 
  \right] 
  ={\underbrace{\int d^{3}x \ 
j(\underline{x},t) \dfrac{1}{L^{3/2}} e^{i\underline{k}\cdot\underline{x}}}_{=\tilde{j}(\underline{k},t)}}
  .
\end{equation}
After using the orthonormality relation \eqref{ortho_relation} this expression reduces to
\begin{equation}\label{eq_for_q}
 \textcolor[RGB]{0,0,204}{
\ddot{\hat{q}}^{H}_{\underline{k}}(t) 
 +
\omega_{\underline{k}}^{2}
 \hat{q}^{H}_{\underline{k}}(t) 
  =\tilde{j}(\underline{k},t)
  ,}
\end{equation}
where  $ \omega_{\underline{k}}^{2} = (\underline{k}^{2}+m^{2})  $ is the energy of the particle in mode $ \underline{k} $.

We now make the assumption that the current vanishes outside a finite time interval,
\begin{equation}
\textcolor[RGB]{0,0,204}{
j(\underline{k},t)\rightarrow 0 \text{ for } t\rightarrow \pm \infty
}.
\end{equation}
As a consequence one can distinguish between early and late times. For early time  Eq.\enskip\eqref{eq_for_q} approaches the homogeneous differential equation. We will call its asymptotic solution by $ \hat{q}^{H}_{\underline{k}}(t) \rightarrow \hat{q}^{in}_{k}(t) $. Explicitly, 
\begin{equation}\label{q_In_operators}
 \textcolor[RGB]{0,0,204}{
 \hat{q}^{in}_{\underline{k}}(t) 
  \approx
  \dfrac{1}{2\omega_{\underline{k}}}\left(
	\hat{a}^{in}_{\underline{k}} 
	e^{-i\omega_{\underline{k}}t}
	+
	\hat{a}^{in \text{\ }\dagger}_{-\underline{k}}  
	e^{i\omega_{\underline{k}}t}
  \right) 
  ,
   \enskip t\rightarrow -\infty,}
\end{equation}
where $ \hat{a}_{\underline{k},in} $ denotes the annihilation operator, whereas $ \hat{a}^{\dagger}_{\underline{k},in} $ is the corresponding creation operator. Their commutator is
\begin{equation}\label{crea_anni_commutator}
\left[ 
\hat{a}^{in}_{\underline{k}}
,
\hat{a}^{in \ \dagger}_{\underline{k'}}
\right] 
=2\omega_{\underline{k}}\delta_{\underline{k},\underline{k'}}
.
\end{equation}
At late times Eq.\enskip\eqref{eq_for_q} also reduces to a homogeneous type. In this case the asymptotic solution $ \hat{q}^{H}_{\underline{k}}(t) \rightarrow \hat{q}^{out}_{k}(t) $ reads
\begin{equation}\label{q_Out_operators}
 \textcolor[RGB]{0,0,204}{
 \hat{q}_{\underline{k},out}(t) 
  \approx
  \dfrac{1}{2\omega_{\underline{k}}}\left(
	\hat{a}^{out}_{\underline{k}} 
	e^{-i\omega_{\underline{k}}t}
	+
	\hat{a}^{out \ \dagger}_{-\underline{k}}  
	e^{i\omega_{\underline{k}}t}
  \right) 
  ,
  \enskip t\rightarrow +\infty,
  }
\end{equation}
where the new operators fulfil a commutation relation similar to Eq.\enskip\eqref{crea_anni_commutator}.
The solution for $ \hat{q}_{\underline{k}}(t) $, at times for which $ j(\underline{x},t) $ is active, would then consist of the homogeneous solution plus a term containing the current:
\begin{equation}\label{q_full}
 \hat{q}^{H}_{\underline{k}}(t) 
  =
  \hat{q}^{in}_{\underline{k}}(t) 
  +
    \dfrac{1}{\omega_{\underline{k}}}
    \int^{t}_{-\infty}
    dt'
    \sin\left[\omega_{\underline{k}}(t-t') \right] \tilde{j}(\underline{k},t')
    ,
\end{equation}
where $ \bar{j_{\underline{k}}}(\omega_{\underline{k}})= \int^{\infty}_{-\infty}dt \tilde{j}(\underline{k},t) e^{i\omega_{\underline{k}}t} $ is the temporal Fourier transform of the current.
For late times $ t\rightarrow +\infty $ the expression above approaches to
\begin{equation}\label{q_Out_by_q_In}
 \textcolor[RGB]{0,0,204}{
\hat{q}^{out}_{\underline{k}}(t) 
  \approx
  \hat{q}^{in}_{\underline{k}}(t) 
  +
    \dfrac{1}{\omega_{\underline{k}}}
    \int^{\infty}_{-\infty}
    dt'
    \sin
    \left[
    \omega_{\underline{k}}(t-t') 
    \right]
     \tilde{j}(\underline{k},t')
  .}
\end{equation}

After splitting the sinus function, we find 
\begin{equation}
\hat{q}^{out}_{\underline{k}}(t) 
  =
  \hat{q}^{in}_{\underline{k}}(t) 
  -
  	\dfrac{i}{2\omega_{\underline{k}}}e^{i\omega_{\underline{k}}t}\
	\bar{j_{\underline{k}}}(-\omega_{\underline{k}})
  +
  	\dfrac{i}{2\omega_{\underline{k}}}e^{-i\omega_{\underline{k}}t}\
  	\bar{j_{\underline{k}}}(\omega_{\underline{k}}),
\end{equation}
From this equation we can obtain the connection between creation and annihilation operators associated with the asymptotically far fields $ t\rightarrow \pm \infty $. In compact notation
\begin{subequations}\label{differ_by_current}
\begin{align}
\hat{a}^{out}_{\underline{k}}=  \hat{a}^{in}_{\underline{k}}+i
%j_k
\bar{j_{\underline{k}}}(\omega_{\underline{k}}) ,
&\\
\hat{a}^{out \ \dagger}_{\underline{k}} = \hat{a}^{in \ \dagger}_{\underline{k}}
-i
%j_k
\bar{j_{\underline{k}}}(-\omega_{\underline{k}})  .
\end{align}
\end{subequations}
This shows that, in the presence of an external current, the two sets of second quantization operators are not the same. Therefore we need to differ between the corresponding $ in $ and $ out $ eigenstates. Particularly, it has to be stated that the vacua also differ in this scenario.% The concept of early and later times to fully solve the full equations will be discussed further in later chapters.

It is important to stress, that the full solution $  \hat{q}^{H}_{\underline{k}}(t) 
 $ found in Eq.\enskip\eqref{q_full} has to be understood in the Heisenberg picture.
 From this we can proceed as shown in section \textbf{2.2}.  
We split the Hamiltonian as done there: $ \hat{H}= \hat{H_0} +\hat{V}_{H} $. 
\begin{equation}
\hat{H}_{0}({\Phi},{\Pi})=
\int d^{3}x 
\,
\left[ 
\frac{1}{2}(\hat{\Pi}^{H})^{2} + \frac{1}{2}(\nabla \hat{ \Phi}^{H})^{2} 
+\frac{1}{2}m^{2}(\hat{\Phi}^{H})^{2}
\right] 
,
\end{equation}
\begin{equation}
\hat{V}_{H}(\Phi)=
\int d^{3}x 
\,
j\hat{\Phi}^{H}
.
\end{equation}
Expressing both field operators in terms of the Fourier basis given in \eqref{box_quanta}, and using the orthonormality relation Eq.\enskip\eqref{ortho_relation}, as well as the reality condition of the field for $ \hat{q}_{-\underline{k}}(t)=\hat{q}^{\star}_{\underline{k}}(t) $ we can express the Hamiltonian as follows:
\begin{equation}
\hat{H}_{0}({q},\dot{{q}})=
\sum_{\underline{k}}
\left\lbrace 
\frac{1}{2}(\dot{\hat{q}}^{H}_{\underline{k}}(t))^{2}
+\frac{1}{2}\omega_{\underline{k}}^{2}(\hat{q}^{H}_{\underline{k}}(t))^{2}
\right\rbrace 
,
\end{equation}
\begin{equation}\label{V_heisenberg_with_q_and_j}
\hat{V}_{H}(q)=
\sum_{\underline{k}}
\tilde{j}(\underline{k},t)\hat{q}^{H}_{\underline{k}}(t).
\end{equation}
From this form we go to the Interaction picture. In the present context, the potential $ V_{I} $ appearing in Eq.\enskip\eqref{eq:time_evo_Omega_interaction} reads:
\begin{equation}\label{V_I_with_q_and_j}
\hat{V}_{I}(q_{I})=
\sum_{\underline{k}}
\tilde{j}(\underline{k},t)\hat{q}^{I}_{\underline{k}}(t),
\end{equation}
where we used Eq.\enskip\eqref{eq:operator_interac_heisenberg} to transform $ \hat{q}_{\underline{k}}(t) $ into the Interaction picture
\begin{equation}
\hat{q}^{I}_{\underline{k}}(t)=
\hat{\Omega}_{I}(t)
\hat{q}^{H}_{\underline{k}}(t)
\hat{\Omega}_{I}^{-1}(t).
\end{equation}
To have a well defined operator $ \hat{\Omega}_{I}(t) $  we need conditions for any $ \hat{\Omega} $ so that $\hat{\Omega} \rightarrow 1 $ as stated for the Interaction picture in Eq.\enskip\eqref{eq:time_evo_Omega_interaction} which is at the moment mostly depended on the current $ j $.
%
%
%
%%
%
%The effect of these conditions on Eq.\enskip\eqref{eq:time_evo_Omega_interaction} leads to 
%\begin{equation}
%\textcolor[RGB]{0,0,204}{
%\Omega_{I}(\pm \infty) \rightarrow 1
%}
%\end{equation}
%%This means 
The early time condition at $ t=-\infty $ defines the $ in $ picture in reminiscence to the first asymptotic solution given in Eq\enskip\eqref{q_In_operators} and it  writes:
\begin{equation}\label{eq:time_evo_Omega_in}
\textcolor[RGB]{0,0,204}{
	i
	\frac{\partial}{\partial t}
	\hat{\Omega}_{in}(t)
=
	\hat{V}_{in}(t)
	\hat{\Omega}_{in}(t)
	,
	}
	\end{equation}
 where the initial condition $	\hat{\Omega}_{in}(-\infty)=1$ has to be fulfilled. Contrary to the previous case the operator of the $ out $ picture will satisfy the differential equation:
\begin{equation}\label{eq:time_evo_Omega_out}
\textcolor[RGB]{0,0,204}{
	i
	\frac{\partial}{\partial t}
	\hat{\Omega}_{out}(t)
	=
	\hat{V}_{out}(t)
	\hat{\Omega}_{out}(t)
	,
  }
\end{equation}
with $\hat{\Omega}_{out}(+\infty)=1$.
%The connecting operator between the pictures is called scattering operator or S-Matrix. It is defined as :\footnote{We will now go to natural units $c= \hbar=1 $ and drop the operator hat for convenience sake}
%\begin{equation}\label{eq:S_in_out}
%\textcolor[RGB]{0,0,204}{
%	S
%	%=\Omega_{in}(t)\Omega_{out}^{-1}(t)
%	=\Omega_{I}(\infty)
%	.}
%\end{equation}
\chapter{Scattering operator}\label{Scattering operator}
\section{Solutions for the Interaction, $ \pmb{in} $  and $ \pmb{out} $ picture}\label{solutions_interaction_in_out}
In this section we solve the differential equations for the various pictures established in section \ref{Interactionpicture} and \ref{in_out_picture_external_currents}.
We start with the Interaction picture depended on $ t' $  and integrate both sides of Eq.\enskip\eqref{eq:time_evo_Omega_interaction}. For $ t>0 $ its left-hand side gives:
\begin{equation}\label{first_term_left_Omega_time_order}
\int_{0}^{t}\mathrm{d}t'
 i 
 \frac{\partial}{\partial_{t'}} 
 \hat{\Omega}_{I} (t')
 =
 i
 \left[ 
\hat{\Omega}_{I}(t) -1
 \right] 
 ,
\end{equation}
where the initial condition $ \hat{\Omega}_{I}(0)=1 $ has been used.
With this formula and the integral over the right-hand side of \eqref{eq:time_evo_Omega_interaction}, we find an expression for $ \hat{\Omega}_{I}(t) $.
\begin{equation}
\hat{\Omega}_{I}(t)=
1
-
i
\int_{0}^{t}\mathrm{d}t'\hat{V}_{I}(t')\hat{\Omega}_{I}(t')	,
\end{equation}
since the expression has an $ \hat{\Omega}_{I}(t) $ on the other side we will go on by an iterative approach.
\begin{equation}\label{Omega_first_terms}
\begin{split}
\hat{\Omega}_{I}(t)
&=
1
-
i 
\int_{0}^{t}\mathrm{d}t'\hat{V}_{I}(t')
\cdot
\left( 
1
-
i
\int_{0}^{t'}\mathrm{d}t''\hat{V}_{I}(t'')\hat{\Omega}_{I}(t'')
\right) 
\\
&=
1
-
i
\int_{0}^{t}\mathrm{d}t'\hat{V}_{I}(t')
+i^{2} 
\int_{0}^{t}\mathrm{d}t'
\int_{0}^{t'}\mathrm{d}t''
\hat{V}_{I}(t'')\hat{\Omega}_{I}(t'')
.
\end{split}
\end{equation}
The iteration increments the power of $ i $ and the number of integrals. By repeating the operation described above we can write
\begin{equation}\label{Omega_different_t}
\hat{\Omega}_{I}(t) =
\sum\limits_{n=0}^{\infty} 
(-i)^{n}
\int_{0}^{t}\mathrm{d}t_1\int_{0}^{t_{1}}\! \! \mathrm{d}t_2
 \ldots
 \int_{0}^{t_{n-1}}\! \! \mathrm{d}t_n
  \hat{V}_{I}(t_1)\cdot \ldots \cdot \hat{V}_{I}(t_n)
  .
\end{equation}
A problematic aspect of this series are the different integral limits. Each term introduces a new $ t_{i} $ and keeps the previous $ t_{i-1} $ as an integral variable which forces us to solve them in a strict order. 
%\footnote{Proofs and elaborations for time ordering are to be found in the Appendices }
To circumvent this formal aspect we will perform some additional operations.
Let us consider the term  from  Eq.\enskip\eqref{Omega_different_t} containing the product of two interactions :
\begin{equation}\label{regular_function}
I(t)=
\int_{0}^{t}\mathrm{d}t_1\int_{0}^{t_1}\! \! \mathrm{d}t_2
\hat{V}_{I}(t_1)\hat{V}(t_2)
.
\end{equation}
By developing the change of variable $ t_{2} \longleftrightarrow t_{1} $,\footnote{The Jacobian of this change of variable is the unity} this integral can be written as
\begin{equation}\label{regular_function_changed}
I(t)=
\int_{0}^{t}\mathrm{d}t_2\int_{0}^{t_2}\! \! \mathrm{d}t_1
\hat{V}_{I}(t_2)\hat{V}_{I}(t_1)
.
\end{equation}
We find an alternative representation of $ I(t) $ by adding \eqref{regular_function} and \eqref{regular_function_changed}:
\begin{equation}\label{I_by_adding}
I(t)=
\dfrac{1}{2}
	\int_{0}^{t}\mathrm{d}t_1\int_{0}^{t_1}\! \! \mathrm{d}t_2
			\hat{V}_{I}(t_1)\hat{V}_{I}(t_2)
+
\dfrac{1}{2}
	\int_{0}^{t}\mathrm{d}t_2\int_{0}^{t_2}\! \! \mathrm{d}t_1
			\hat{V}_{I}(t_2)\hat{V}_{I}(t_1).
\end{equation}
In order to have a common integration limit $ t $, we introduce the chronological time ordering.
\begin{equation}\label{eq:chron-time_ordering}
\textcolor[RGB]{0,0,204}{
T(\hat{V}_{I}(t_1), \hat{V}_{I}(t_2))=\hat{V}_{I}(t_1)\ \hat{V}_{I}(t_2)\ \theta (t_1 -t_2)\ +\ \hat{V}_{I}(t_2)\  \hat{V}_{I}(t_1) \ \theta (t_2-t_1)
.}
\end{equation}
The chronological time ordering sets operators depending of earlier times to the right and later to the left. The the Heaviside-Step-function is $ 0 $ for negative values of its argument and $ 1 $ when it becomes positive.
By subtracting $ t_1 $ and $ t_2 $ in the argument of the step functions we are able to switch between the two terms in Eq.\enskip\eqref{I_by_adding} and extending the integral limits to $ t $, since it sets terms to zero for negative arguments. Therefore no change appears in the result of the integral by extending the limit. We used
for $ t_1 > t_2  \rightarrow \theta (t_1 -t_2)$ $\thinspace $
and
for $ t_2 > t_1  \rightarrow \theta (t_2 -t_1)$. 
By applying Eq.\enskip\eqref{eq:chron-time_ordering} at Eq.\enskip\eqref{I_by_adding}, we find the desired notation:
\begin{equation*}
\begin{split}
&I(t)=\dfrac{1}{2}
\int_{0}^{t}\mathrm{d}t_1\int_{0}^{t}\! \! \mathrm{d}t_2
\hat{V}_{I}(t_1)\ \hat{V}_{I}(t_2)\ \theta (t_1 -t_2)
\\
&\qquad\qquad\qquad\qquad\qquad\qquad
+
\dfrac{1}{2}
\int_{0}^{t}\mathrm{d}t_1\int_{0}^{t}\! \! \mathrm{d}t_2
\ \hat{V}_{I}(t_2)\  \hat{V}_{I}(t_1) \ \theta (t_2-t_1)
,
\end{split}
\end{equation*}
which can be compactly written in the following form.
\begin{equation}
I(t)=\dfrac{1}{2!}
\int_{0}^{t}\mathrm{d}t_1\int_{0}^{t}\! \! \mathrm{d}t_2
T(\hat{V}_{I}(t_1),\hat{V}_{I}(t_2))
.
\end{equation}

This case of two interactions is generalized to terms involving $\hat{V}(t) $ $ n $-times in the appendix \ref{chronological_time}.  
Applying the $ T $ operator allows us to write the solution to $ \hat{\Omega}_{I}(t) $ given in Eq.\enskip\eqref{Omega_different_t} in the time-ordered form:
\begin{equation}\label{Omega_I_long_timeorderd}
\hat{\Omega}_{I}(t)
=
\frac{(-i)^{n}}{n!}
\int_{0}^{t}\mathrm{d}t_1\int_{0}^{t}\! \! \mathrm{d}t_2
 \ldots
 \int_{0}^{t}\! \! \mathrm{d}t_n
 T\left\lbrace \hat{V}_{I}(t_1), \ldots , \hat{V}_{I}(t_n)\right\rbrace 
 .
\end{equation}
Observe that this expression is a non-pertubative result, which can  be written in a symbolically notation
\begin{equation}\label{eq:Omega_I_Chron_0}
\textcolor[RGB]{0,0,204}{
\hat{\Omega}_{I}(t)
=T\left( e^{-i\int_{0}^{t}\mathrm{d}t^{\prime} \hat{V}_{I}(t^{\prime})} \right)
	\! ,\text{\enskip for  }  t\geq 0 
	.}
\end{equation}
To not limit the Interaction picture only to positive $ t $ values, we need a complementary expression for negative $ t $'s. Assuming $ t<0 $, the integral in Eq.\enskip\eqref{first_term_left_Omega_time_order} is changed to:
 \begin{equation}\label{first_term_left_anti_chrono}
 \int_{t}^{0}\mathrm{d}t'
 i
 \frac{\partial}{\partial_{t'}} 
 \hat{\Omega}_{I} (t')
 =
 i
 \left[ 
1 -\hat{\Omega}_{I}(t)
 \right] .
 \end{equation}
 Alongside performing in the integral of the right-hand side of  Eq.\enskip\eqref{eq:time_evo_Omega_interaction} in the new limits, we find:
 \begin{equation}
  \hat{\Omega}_{I}(t)=
1
+
i
\int^{0}_{t}\mathrm{d}t'\hat{V}_{I}(t')\hat{\Omega}_{I}(t')	.
  \end{equation} 
From this, our infinite sum expression still holds up to a different sign:
\begin{equation}\label{Omega_anti_without_anti}
\hat{\Omega}_{I}(t) =
\sum\limits_{n=0}^{\infty} 
i^{n}
\int^{0}_{t}\mathrm{d}t_1\int^{0}_{t_{1}}\! \! \mathrm{d}t_2
 \ldots
 \int^{0}_{t_{n-1}}\! \! \mathrm{d}t_n
  \hat{V}_{I}(t_1)\cdot \ldots \cdot \hat{V}_{I}(t_n).
\end{equation}
The key difference now stands in the negativity of all $ t $ and a logical order for them would prefer later times to the right, coming closer to $ 0 $. This requires the anti-chronological time ordering:
 \begin{equation}\label{eq:anti-chron-time_ordering}
\textcolor[RGB]{0,0,204}{
\bar{T}(\hat{V}(t_1), \hat{V}(t_2))=\hat{V}(t_2)\ \hat{V}(t_1)\ \theta (t_1 -t_2)\ +\ \hat{V}(t_1)\  \hat{V}(t_2) \ \theta (t_2-t_1)
.}
\end{equation}
A generalized expression containing the product of several $ \hat{V}(t) $'s is given in the appendix \ref{chronological_time}.
Using it similar as before:
\begin{equation}
\hat{\Omega}_{I}(t) =
\sum\limits_{n=0}^{\infty} 
\frac{i^{n}}{n!}
\int^{0}_{t}\mathrm{d}t_1\int^{0}_{t}\! \! \mathrm{d}t_2
 \ldots
 \int^{0}_{t}\! \! \mathrm{d}t_n
 \bar{T}\left\lbrace \hat{V}_{I}(t_1), \ldots , \hat{V}_{I}(t_n)\right\rbrace .
\end{equation}
As a consequence, we obtain a second expression:\begin{equation}\label{eq:Omega_I_Chron1_0<}
\hat{\Omega}_{I}(t)
= \bar{T}\left( e^{i\int^{0}_{t}\mathrm{d}t^{\prime} \hat{V}_{I}(t^{\prime})} \right)
	\! ,\text{\enskip for  }  t<0 
	.
\end{equation}
%Often the integration borders are flipped to have a the same sign as Eq.\enskip\eqref{eq:Omega_I_Chron_0}
%\begin{equation}\label{eq:Omega_I_Chron_0<}
%\textcolor[RGB]{0,0,204}{
%\hat{\Omega}_{I}(t)
%= \bar{T}\left( e^{-i\int_{0}^{t}\mathrm{d}t^{\prime} \hat{V}_{I}(t^{\prime})} \right)
%	\! ,\text{\enskip for  }  t<0 
%	.}
%\end{equation}
A notation for $ \hat{\Omega}_{I}(t) $ without specifying the values of $ t $ can be derived by using Heaviside-Step-functions:
\begin{equation}\label{Omega_i_complete}
\hat{\Omega}_{I}(t)
=T\left( e^{-i\int_{0}^{t}\mathrm{d}t^{\prime} \hat{V}_{I}(t^{\prime})} \right)
\theta(t)
+
 \bar{T}\left( e^{+i\int_{t}^{0}\mathrm{d}t^{\prime} \hat{V}_{I}(t^{\prime})} \right)
 \theta(-t)
 .
\end{equation}

For the $ in $ picture we proceed in an almost identical fashion to the Interaction picture for $ t > 0 $. Only the lower boundary in the integral is changed to $ -\infty $ as it is the asymptotic condition of this picture. This resolves the need for a two term solution. 
After resumation, we obtain:
\begin{equation}\label{eq:Omega_in_converg}
\textcolor[RGB]{0,0,204}{
\hat{\Omega}_{in}(t)
= T\left( e^{-i\int_{-\infty}^{t}\mathrm{d}t^{\prime} \hat{V}_{in}(t^{\prime})} \right)
	.}
\end{equation}
The $ out $ picture on the other hand follows the derivation of the expression for $ t < 0 $. We start at Eq.\enskip\eqref{first_term_left_anti_chrono} with $ \infty $ instate of $ 0 $. Here we argue $ t $ being smaller then $ \infty $ needs one change of sign like before and anti-chronological ordering $ \bar{T} $ introduced in Eq.\enskip\eqref{eq:anti-chron-time_ordering}, since $ t $ only coming closer to the limit as it runs.
\begin{equation}
\hat{\Omega}_{out}(t) =
\sum\limits_{n=0}^{\infty} 
\frac{(i)^{n}}{n!}
\int^{\infty}_{t}\mathrm{d}t_1\int^{\infty}_{t}\! \! \mathrm{d}t_2
 \ldots
 \int^{\infty}_{t}\! \! \mathrm{d}t_n
 \bar{T}\left\lbrace \hat{V}_{out}(t_1), \ldots , \hat{V}_{out}(t_n)\right\rbrace .
\end{equation}
Performing a flip in the integral limits we conclude:
\begin{equation}\label{eq:Omega_out_converg}
\textcolor[RGB]{0,0,204}{
\hat{\Omega}_{out}(t)
= \bar{T}\left( e^{-i\int_{\infty}^{t}\mathrm{d}t^{\prime} \hat{V}_{in}(t^{\prime})} \right)
	.}
\end{equation}
\section{Representation of the Scattering operator}\label{Connections}
In section \ref{in_out_picture_external_currents} we saw that the creation and annihilation operators associated with the $ in $ and $ out $ pictures are related between each other in Eq.\enskip\eqref{differ_by_current} via the external current. 
We wish to establish how this connection manifests at the level of the corresponding scattering states. To this end, we particularize Eq.\enskip\eqref{eq:state_interac_schrodinger} to the case in which the initial condition is taken at $ t \rightarrow \pm \infty $.
%We start with a state in the Heisenberg picture and do a transformation to the $ in $ picture following the same way as Eq.\enskip\eqref{eq:state_interac_schrodinger} for the Interaction picture.
If the initial condition is $ t \rightarrow - \infty $, we have
\begin{equation}\label{in_state_from_Heisenberg}
\ket{\Psi_{in}(t)}=\hat{\Omega}_{in}(t)\ket{\Psi_{H}}.
\end{equation}
Likewise for the $ out $ picture we write:
\begin{equation}\label{out_state_from_Heisenberg}
\ket{\Psi_{out}(t)}=\hat{\Omega}_{out}(t)\ket{\Psi_{H}}.
\end{equation}
Combining both Eq.\enskip\eqref{in_state_from_Heisenberg} and Eq.\enskip\eqref{out_state_from_Heisenberg} we find
\begin{equation}\label{in_to_out_state}
\ket{\Psi_{in}(t)}=\hat{\Omega}_{in}(t)\hat{\Omega}^{-1}_{out}(t)\ket{\Psi_{out}(t)}.
\end{equation}
The product of $ \hat{\Omega} $'s defines the scattering operator $ \hat{S} $:% It connects any state 
%\begin{equation}
%\ket{in}=\hat{S}\ket{out}.
%\end{equation}
\begin{equation}\label{S_defi}
\textcolor[RGB]{0,0,204}{
	\hat{S}=
	\hat{\Omega}_{in}(t)\hat{\Omega}^{-1}_{out}(t)
	.
}
\end{equation}
We will verify explicitly that Eq.\enskip\eqref{S_defi} is  time independent. For this we take the partial derivative of $ \hat{S} $ with respect to $ t $. This operation leads to
\begin{equation}\label{partial_t_for_indepen}
\begin{split}
i\partial_{t}
\left( \hat{\Omega}_{in}(t)\hat{\Omega}^{-1}_{out}(t) \right)
&= i\dot{\hat{\Omega}}_{in}(t)\hat{\Omega}^{-1}_{out}(t)
 	+ 
 		i\hat{\Omega}_{in}(t)\dot{\hat{\Omega}}^{-1}_{out}(t)
\\
&= \hat{V}_{in}(t)\hat{\Omega}_{in}(t)\hat{\Omega}^{-1}_{out}(t)
	+ 	
		i\hat{\Omega}_{in}(t)\dot{\hat{\Omega}}^{-1}_{out}(t)
%\\
%&= \hat{V}_{in}(t)\hat{\Omega}_{in}(t)\hat{\Omega}^{-1}_{out}(t)
%	+ 	
%		\hat{\Omega}_{in}(t)(\hat{\Omega}_{out}(t) (-\hat{V}_{out}(t)))^{-1}	
.
\end{split}
\end{equation}
In the first term of the second line we have used Eq.\enskip\eqref{eq:time_evo_Omega_in}. The evaluation of $ \dot{\hat{\Omega}}^{-1}_{out}(t) $ requires some additional steps. We start by differentiating the identity $ 
\hat{\Omega}_{out}(t)\hat{\Omega}^{-1}_{out}(t)  = 1 $. We then find the relation
\begin{equation}
\begin{split}
		i\hat{\Omega}_{out}(t)\dot{\hat{\Omega}}^{-1}_{out}(t)
	&=
		-i\dot{\hat{\Omega}}_{out}(t)\hat{\Omega}^{-1}_{out}(t)
\end{split}
\end{equation}
Afterwards, we multiply the left by $ \hat{\Omega}_{out}(t) $ and use Eq.\enskip\eqref{eq:time_evo_Omega_out} in the right-hand side. As a consequence, 
\begin{equation}
\begin{split}
				i\dot{\hat{\Omega}}^{-1}_{out}(t)
			&=	
					-\hat{\Omega}^{-1}_{out}(t)\hat{V}_{out}(t)
.
\end{split}
\end{equation}
This results is inserted in Eq.\enskip\eqref{partial_t_for_indepen}:
\begin{equation}
\begin{split}
i\partial_{t}
\left( \hat{\Omega}_{in}(t)\hat{\Omega}^{-1}_{out}(t) \right)
&=  \hat{V}_{in}(t)\hat{\Omega}_{in}(t)\hat{\Omega}^{-1}_{out}(t)
	- 	
		\hat{\Omega}_{in}(t)\hat{\Omega}^{-1}_{out}(t)\hat{V}_{out}(t)
\end{split}
\end{equation}
A picture transformation like stated for the Interaction picture in Eq.\enskip\eqref{eq:operator_interac_heisenberg} allows us to write both potentials in the Heisenberg representation.
\begin{equation}\label{S_t_indep}
\begin{split}
i\partial_{t}
\hat{S}
&= \hat{\Omega}_{in}(t)\hat{V}_{H}\hat{\Omega}^{-1}_{out}(t)
+
\hat{\Omega}_{in}(t)(-\hat{V}_{H})\hat{\Omega}^{-1}_{out}(t) 
=0
.
\end{split}
\end{equation}
This time independence of the product gives us the choice to set $ t $ to any value. In particular by setting it to $ t=\infty $ we find that Eq.\enskip\eqref{S_defi} reduces to 
\begin{equation}\label{S_first_3_1}
\textcolor[RGB]{0,0,204}{
\begin{split}
\hat{S} 
%&= \hat{\Omega}_{in}(t)\hat{\Omega}^{-1}_{out}(t)
%\\
&=\hat{\Omega}_{in}(\infty)
%\\
%&=\hat{\Omega}^{-1}_{out}(-\infty)
,
\end{split}
}
\end{equation}
where the initial conditions, $\hat{\Omega}_{out}(\infty) = 1  $  has been used. Similarly, if $ t=\infty $ we use $ \hat{\Omega}_{in}(-\infty)=1 $ and
\begin{equation}\label{S_first_3_2}
\textcolor[RGB]{0,0,204}{
\begin{split}
\hat{S}
%&= \hat{\Omega}_{in}(t)\hat{\Omega}^{-1}_{out}(t)
%\\
%&=\hat{\Omega}_{in}(\infty)
%\\
&=\hat{\Omega}^{-1}_{out}(-\infty)
,
\end{split}
}
\end{equation}

Taking into account, that a lot of literature around quantum field theory relates the $ \hat{S} $ operator in term of the Interaction picture, we shall verify a secondary set of relations. Starting in a state in the $ in $ picture, we go to the Heisenberg and then to the Interaction picture similar to what we have done in Eq.\enskip\eqref{S_defi}.
\begin{equation}
\begin{split}
\ket{\Psi_{in}(t)}&=\hat{\Omega}_{in}(t)\ket{\Psi_{H}}
\\
&=\hat{\Omega}_{in}(t)\hat{\Omega}^{-1}_{I}(t)\ket{\Psi_{I}(t)}
.
\end{split}
\end{equation}
This new product of unitary operators is also time independent. This can be shown by taking the partial derivative as seen in Eq.\enskip\eqref{partial_t_for_indepen}. The derivative of the inverse operator $ \left[ \dot{\hat{\Omega}}^{-1}_{I}(t)\right]  $ will also result in a $ -\hat{V}_{H}\hat{\Omega}^{-1}_{I}(t) $. Thus,
\begin{equation}\label{product_2_t_indep}
\begin{split}
i\partial_{t}
\hat{\Omega}_{in}(t)\hat{\Omega}^{-1}_{I}(t)
&= \hat{\Omega}_{in}(t)\hat{V}_{H}\hat{\Omega}^{-1}_{I}(t)
+
\hat{\Omega}_{in}(t)(-\hat{V}_{H})\hat{\Omega}^{-1}_{I}(t)
=0
.
\end{split}
\end{equation}
%\begin{subequations}
%\textcolor[RGB]{0,0,204}{
%\begin{align}
%	\hat{\Omega}_{in}(t)\hat{\Omega}_{I}(t)^{-1}
%	&=\hat{S}\ \hat{\Omega}_{I}(\infty)^{-1}
%	&\\
%	&=
%	\hat{\Omega}_{in}(0)	.
%\end{align}
%}
%\end{subequations} 
To bring the $ \hat{S} $ operator into our equation, we substitute $ \hat{\Omega}_{in}(t) $ using the definition in Eq.\enskip\eqref{S_defi}:
\begin{equation}
\begin{split}
\hat{\Omega}_{in}(t)\hat{\Omega}^{-1}_{I}(t)
=\hat{S}\hat{\Omega}_{out}(t)\hat{\Omega}^{-1}_{I}(t)
.
\end{split}
\end{equation}
 We can set $t $ to the initial condition of a picture now including $ \hat{\Omega}_{I}(0) =1 $, since we have verified time independence.
\begin{equation}\label{second_set_S_relations}
\textcolor[RGB]{0,0,204}{
\begin{split}
\hat{\Omega}_{in}(t)\hat{\Omega}_{I}^{-1}(t)
&=\hat{S}\ \hat{\Omega}_{I}^{-1}(\infty)
\\
&=
	\hat{\Omega}_{in}(0)	
.
\end{split}
}
\end{equation}
Finally, we want to derive the following expression:
\begin{equation}\label{Omeaga_in_for_GML}
\textcolor[RGB]{0,0,204}{
\hat{\Omega}_{in}(t)=
\bar{T}
\left( 
 e^{i\int_{t}^{\infty}\mathrm{d}t^{\prime} \hat{V}_{in}(t^{\prime})}
\right) 
\hat{S}.}
\end{equation}
This formula is essential in our way of establishing the Gell-Mann Low formula, which is carried out in the next section. First it must satisfy the differential equation for $ \hat{\Omega}_{in}(t) $ [see Eq.\enskip\eqref{eq:time_evo_Omega_in}]. To reduce the number of integrals directly depending on $ t $ to one, we use the representation for the exponential function without anti-chronological time ordering similar to Eq.\enskip\eqref{Omega_anti_without_anti}:
\begin{subequations}
\begin{align}
i\partial_{t}\hat{\Omega}_{in}(t)
	&=i\partial_{t} \left( 
	\sum_{n} i^{n}
  	   \int_{t}^{\infty}\mathrm{d}t_1 \hat{V}_{in}(t_1)
		\ldots    
	    \int_{t_{n-1}}^{\infty}\mathrm{d}t_n \hat{V}_{in}(t_n)
		\right)
		\hat{S}
		.
	\end{align}
%	\text{by making an integration by part we find,}
%	\begin{align}
%	i\partial_{t}\hat{\Omega}_{in}(t)&=i \left( 
%	\sum_{n} i^{n}
%		\partial_{t}
%		\left[
%		\bar{\hat{V}}_{in}(\infty)-\bar{\hat{V}}_{in}(t)
%		 \right] 
%  	    \int_{t_1}^{\infty}\mathrm{d}t_2 \hat{V}_{in}(t_2)
%		\ldots    
%	    \int_{t_{n-1}}^{\infty}\mathrm{d}t_n \hat{V}_{in}(t_n)
%		\right)
%		\hat{S}.
%\end{align}
\text{The Leibniz integral rule can be applied to obtain,}
\begin{align}
	i\partial_{t}\hat{\Omega}_{in}(t)
	&= i\left( 
	\sum_{n} i^{n}
		\left(-
		\hat{V}_{in}(t)
		 \right) 
  	     \int_{t_1}^{\infty}\mathrm{d}t_2 \hat{V}_{in}(t_2)
		\ldots    
	    \int_{t_{n-1}}^{\infty}\mathrm{d}t_n \hat{V}_{in}(t_n)
		\right)
		\hat{S}
		.	
\end{align}
\end{subequations}
An $ i $ taken out of the sum allows us to remove the negative sign and restores the correct power $ n-1 $. By using the definition of anti-chronological time ordering (see appendix \ref{chronological_time}) and changing the summation index $\left(  n-1 = \tilde{n} \right) $,
\begin{equation}
	i\partial_{t}\hat{\Omega}_{in}(t)	=\hat{V}_{in}(t)
	\sum_{\tilde{n}=1}^{\infty}
	\frac{i^{\tilde{n}}}{\tilde{n}!} 
 	     \int_{t^{\prime}_0}^{\infty}\mathrm{d}t^{\prime}_1 
		\ldots    
	   \int_{t^{\prime}_{0}}^{\infty}\mathrm{d}t^{\prime}_{\tilde{n}}
		\bar{T}
		\left( 
		\hat{V}_{in}(t^{\prime}_1)
		\ldots
		     \hat{V}_{in}(t^{\prime}_{\tilde{n}})
		\right)\hat{S}
		.
\end{equation}
%We again write the expression in a compact form and
%since $ t $ being the earliest time in the integral and anti-chronological time ordering can be applied we are allowed to keep $ \hat{V}_{in}(t) $ to the left and out of the $ \bar{T} $ operator.
\begin{equation}
\begin{split}
	i\partial_{t}\hat{\Omega}_{in}(t)
	&=\hat{V}_{in}(t)	
\underbrace{	
	\bar{T}
	\left( 
	 e^{i\int_{t}^{\infty}\mathrm{d}t^{\prime} \hat{V}_{in}(t^{\prime})}
	\right)\hat{S}
}_{=\hat{\Omega}_{in}(t)}	
	.
\end{split}
\end{equation}
 Now we verify that the initial condition $\left[  \hat{\Omega}_{in}(-\infty)= 1 \right] $ still holds for the representation given in Eq.\enskip\eqref{Omeaga_in_for_GML}. For this we evaluate it as follows:
%\begin{subequations}
%\begin{align}
%\hat{\Omega}_{in}(t)
%	&=\bar{T}
%	\left( 
%	 e^{i\int_{-\infty}^{\infty}\mathrm{d}t^{\prime} \hat{V}_{in}(t^{\prime})}
%	\right) \hat{S}
%	&\\
%	&=
%	(\hat{\Omega}_{in}(\infty))^{\dagger}\hat{S}
%	&\\
%	&=\hat{S}^{\dagger}\hat{S}=\hat{S}^{-1}\hat{S}=1
%\end{align}
%\end{subequations}
\begin{equation}\label{Omega_in_minus_infty_GML}
\hat{\Omega}_{in}(-\infty)
=\bar{T}
	\left( 
	 e^{i\int_{-\infty}^{\infty}\mathrm{d}t^{\prime} \hat{V}_{in}(t^{\prime})}
	\right) \hat{S}
\end{equation}
At this point we need to make use of an important property of the time ordered operators. 

The hermitian conjugation of anti-chronological time ordering of a product of operators $ \hat{V}(t)$ turns it into the chronological time ordering of same operators, as long as $ \hat{V}(t) $ is hermitian. Note that this can be seen immediately by taking the hermitian conjugate of Eq.\enskip\eqref{eq:anti-chron-time_ordering}:
\begin{equation}\label{barT_to_T}
\left[
\bar{T}(\hat{V}(t_{1})\ldots\hat{V}(t_{n}))
\right]^{\dagger}
=T(\hat{V}(t_{1})\ldots\hat{V}(t_{n}))
.
\end{equation}
%Observe that the operators positions are switched. Yet the Heaviside-step-functions are unchanged which gives directly the definition in Eq.\enskip\eqref{eq:chron-time_ordering}.
%
By keeping this in mind, we can recognize in Eq.\enskip\eqref{Omega_in_minus_infty_GML} the hermitian conjugate of Eq.\enskip\eqref{eq:Omega_in_converg} evaluated at $ t=\infty $:
\begin{equation}
\begin{split}
\hat{\Omega}_{in}(-\infty)
&= 	(\hat{\Omega}_{in}(\infty))^{\dagger}\hat{S}
	\\
	&\!\!\!\!\!\underbrace{=}_{\text{Eq.\enskip\eqref{S_first_3_1}}}\! \hat{S}^{-1}\hat{S}=1,
\end{split}
\end{equation}
where the last step relies on the unitary of $ \hat{S} $. This can be seen straight away  by taking the hermitian conjugate of Eq.\enskip\eqref{S_defi}.
\section{Gell-Mann Low formula}
%Recalling Eq.\enskip\eqref{eq:operator_interac_heisenberg} we can write:
%\begin{equation}\textcolor[RGB]{0,0,204}{
%\hat{q}^{H}_{\underline{k}}(t)=\hat{\Omega}_{I}(t)^{-1}\hat{Q}_{I}(t)\ \hat{\Omega}_{I}(t).
%}
%\end{equation}
Let us consider Eq.\enskip\eqref{eq:operator_interac_heisenberg} with the initial condition taken at $ t \rightarrow - \infty $:
\begin{equation}\textcolor[RGB]{0,0,204}{
\hat{q}^{H}_{\underline{k}}(t)=\hat{\Omega}_{in}^{-1}(t)\hat{q}^{in}_{\underline{k}}(t)\ \hat{\Omega}_{in}(t).
}
\end{equation}
Expressing $ \hat{\Omega}^{-1}_{in}(t) $ using Eq.\enskip\eqref{Omeaga_in_for_GML} and $ \hat{\Omega}_{in}(t) $ as in Eq.\enskip\eqref{eq:Omega_in_converg}. We find 
\begin{equation}
\hat{q}^{H}_{\underline{k}}(t)
=\left[ 
\bar{T}
\left( 
 e^{i\int_{t}^{\infty}\mathrm{d}t^{\prime} \hat{V}_{in}(t^{\prime})}
\right) 
\hat{S}
\right]^{-1}
%%
\hat{q}^{in}_{\underline{k}}(t)
%%
T
\left( 
 e^{-i\int_{-\infty}^{t}\mathrm{d}t^{\prime} \hat{V}_{in}(t^{\prime})}
\right) 
.
\end{equation}
Since $ \hat{\Omega}_{in}(t) $ is a unitary operator we can replace its inverse by its hermitian conjugate:
\begin{equation}
\begin{split}
\hat{q}^{H}_{\underline{k}}(t)
%&=\left( 
%\bar{T}
%\left( 
% e^{i\int_{t}^{\infty}\mathrm{d}t^{\prime} \hat{V}_{in}(t^{\prime})}
%\right) 
%\hat{S}
%\right)^{\dagger}
%%%
%\hat{q}^{in}_{\underline{k}}(t)
%%%
%T
%\left( 
% e^{-i\int_{-\infty}^{t}\mathrm{d}t^{\prime} \hat{V}_{in}(t^{\prime})}
%\right) 
%\\
&=
\hat{S}^{-1}\left[ 
\bar{T}
\left[ 
 e^{i\int_{t}^{\infty}\mathrm{d}t^{\prime} \hat{V}_{in}(t^{\prime})}
\right) 
\right]^{\dagger}
%%
\hat{q}^{in}_{\underline{k}}(t)
%%
T
\left( 
 e^{-i\int_{-\infty}^{t}\mathrm{d}t^{\prime} \hat{V}_{in}(t^{\prime})}
\right) 
,
\end{split}
\end{equation}
where the unitarity of $ \hat{S} $ has been used.
Taking Eq.\enskip\eqref{barT_to_T} into account, this expression can be written as
\begin{equation}\label{1st_expression_for_Omega_H}
\hat{q}^{H}_{\underline{k}}(t)
=
\hat{S}^{-1} 
T
\left( 
 e^{-i\int_{t}^{\infty}\mathrm{d}t^{\prime} \hat{V}_{in}(t^{\prime})}
\right)
%%
\hat{q}^{in}_{\underline{k}}(t)
%%
T
\left( 
 e^{-i\int_{-\infty}^{t}\mathrm{d}t^{\prime} \hat{V}_{in}(t^{\prime})}
\right) .
\end{equation}
Let us temporary denote 
\begin{equation}
\begin{split}
	\hat{M}(t) = T\left(
	e^{-i\int_{t}^{\infty}\mathrm{d}t^{\prime} \hat{V}_{in}(t^{\prime})}	
	\right)	
	\hat{q}^{in}_{\underline{k}}(t)	
	.
\end{split}
\end{equation}
Observe that the product 
\begin{equation}
\begin{split}
	\hat{M}(t)
T
\left( 
 e^{-i\int_{-\infty}^{t}\mathrm{d}t^{\prime} \hat{V}_{in}(t^{\prime})}
\right)	
	 = 	
	 T
\left( 
\hat{M}(t)
 e^{-i\int_{-\infty}^{t}\mathrm{d}t^{\prime} \hat{V}_{in}(t^{\prime})}
\right) 
	,
\end{split}
\end{equation}
because $ \hat{M}(t) $ is fixed at the latest time (see the integral limits). Therefore,
\begin{equation}
\begin{split}
\hat{q}^{H}_{\underline{k}}(t)
=
\hat{S}^{-1} 
T
\left[ 
T
\left( 
 e^{-i\int_{t}^{\infty}\mathrm{d}t^{\prime} \hat{V}_{in}(t^{\prime})}
\right)
%%
\hat{q}^{in}_{\underline{k}}(t)
%%
 e^{-i\int_{-\infty}^{t}\mathrm{d}t^{\prime} \hat{V}_{in}(t^{\prime})}
\right] 
	.
\end{split}
\end{equation}
Inside of a $ T $-product, the factors commute\footnote{In the case of fermions, one has to apply anti-commutation. Nevertheless, the result stays the same, since the overall signs appearing from fermionic operator repositioning inside the time ordering is linked to $ (-1)^{2n} $, where $n $ is the number of external currents.} and can be written in any order (see appendix \ref{chronological_time}). In particular,
\begin{equation}\label{GML_v0.3}
\begin{split}
\hat{q}^{H}_{\underline{k}}(t)
=
\hat{S}^{-1} 
T
\left[ 
%%
\hat{q}^{in}_{\underline{k}}(t)
%%
T
\left( 
 e^{-i\int_{t}^{\infty}\mathrm{d}t^{\prime} \hat{V}_{in}(t^{\prime})}
\right)
 e^{-i\int_{-\infty}^{t}\mathrm{d}t^{\prime} \hat{V}_{in}(t^{\prime})}
\right] 
	.
\end{split}
\end{equation} 
We want to combine the two exponential functions that remain. In order to achieve this, we first first introduce $\hat{F}(t)
=
 e^{-i\int_{-\infty}^{t}\mathrm{d}t^{\prime} \hat{V}_{in}(t^{\prime})}
$ and consider the operation:
\begin{equation}\label{Moving_f_inside}
\begin{split}
& T
\left( 
 e^{-i\int_{t}^{\infty}\mathrm{d}t^{\prime} \hat{V}_{in}(t^{\prime})}
\right)
\hat{F}(t)
\\
&=
\left\lbrace 
1 
-
i
\int_{t}^{\infty}\mathrm{d}t^{\prime}_{1}
 T
 \left[ 
 \hat{V}_{in}(t^{\prime}_{1})
 \right] 
+
\frac{(-i)^{2}}{2}
\int_{t}^{\infty}\mathrm{d}t^{\prime}_{1}
\int_{t^{\prime}_{1}}^{\infty}\mathrm{d}t^{\prime}_{2}
 T
 \left[ 
 \hat{V}_{in}(t^{\prime}_{1})\hat{V}_{in}(t^{\prime}_{2})
 \right] 
\right\rbrace
\hat{F}(t)
.
\end{split}
\end{equation} 
As $ \hat{F}(t) $ is evaluated at the earliest time the expression above can be written as: 
\begin{equation}\label{Move_in_T_from_right}
\begin{split}
&
\hat{F}(t)\!
-\!
i\!
\int_{t}^{\infty}\! \!\mathrm{d}t^{\prime}_{1}
 T
 \left[ 
 \hat{V}_{in}(t^{\prime}_{1})
 \hat{F}(t)
 \right] \!
+\!
\frac{(-i)^{2}}{2}\!\!
\int_{t}^{\infty}\!  \mathrm{d}t^{\prime}_{1}
\int_{t^{\prime}_{1}}^{\infty}\!\!\mathrm{d}t^{\prime}_{2}
 T
 \left[ 
 \hat{V}_{in}(t^{\prime}_{1})\hat{V}_{in}(t^{\prime}_{2})
\hat{F}(t) 
 \right] 
+\ldots
\\
&=
T\left( 
 e^{-i\int_{t}^{\infty}\mathrm{d}t^{\prime} \hat{V}_{in}(t^{\prime})}
\hat{F}(t)
\right)
.
\end{split}
\end{equation} 
Substituting this result in Eq.\enskip\eqref{GML_v0.3}, we find:
\begin{equation}
\begin{split}
\hat{q}^{H}_{\underline{k}}(t)
=
\hat{S}^{-1} 
T
\left[
%%
\hat{q}^{in}_{\underline{k}}(t)
%%
T
\left( 
 e^{-i\int_{t}^{\infty}\mathrm{d}t^{\prime} \hat{V}_{in}(t^{\prime})}
 e^{-i\int_{-\infty}^{t}\mathrm{d}t^{\prime} \hat{V}_{in}(t^{\prime})}
\right) 
\right] 
.
\end{split}
\end{equation}
%%
%%%
%We see that each term in the series fulfils the condition to be moved into the inner time ordering. In addition the external time ordering would made it possible to commute all $ \hat{V}_{in}(t^{\prime}_{i}) $ if necessary. Inside the inner $ T $ operator we write the compact notation 
To merge both exponential functions the BCH-formula must be applied but due to the chronological time ordering, we could set the appearing commutator to zero. Hence
\begin{equation}\label{3rd_expression_for_Omega_H}
\hat{q}^{H}_{\underline{k}}(t)
=
\hat{S}^{-1} 
T
\left[
%%
\hat{q}^{in}_{\underline{k}}(t)
%%
T
\left( 
\underbrace{
 e^{-i\int_{t}^{\infty}\mathrm{d}t^{\prime} \hat{V}_{in}(t^{\prime})-i\int_{-\infty}^{t}\mathrm{d}t^{\prime} \hat{V}_{in}(t^{\prime})}
 }_{
 =\exp(-i\int_{-\infty}^{\infty}\mathrm{d}t^{\prime} \hat{V}_{in}(t^{\prime}))
 }
\right) 
\right]
.
\end{equation}
We can identify the expression for the $ \hat{S} $-operator given in the second line of Eq.\enskip\eqref{S_first_3_1}. Therefore,
%as long as we keep the scattering operator inside the time ordering, since several non trivial commutations have been performed in the derivation. These are only ensured to be equal to $ 1 $ inside of the chronological time ordering. 
 \begin{equation}\label{Gell_Mann-Low_for_one_Q}
\begin{split}
 \textcolor[RGB]{0,0,204}{
\hat{q}^{H}_{\underline{k}}(t)
=
\hat{S}^{-1} 
T
\left[ 
\hat{q}^{in}_{\underline{k}}(t)
\hat{S}
\right]
.
}
\end{split}
\end{equation}
We remark that, even though the $ \hat{S} $-operator is time independent, it has to remain inside of the time ordering. The reasons for this are commutators involved in the BCH-formula and repositioning $ \hat{q}^{H}_{\underline{k}}(t) $. Only the overall time ordering ensures them to be zero.

Next, consider the case with more than one $ \hat{q}^{H}_{\underline{k}} $-operator. We begin with two operators in the Heisenberg-picture on left side in chronological time ordering.
% and apply Eq.\enskip\eqref{Gell_Mann-Low_for_one_Q} to each one. This results in
\begin{equation}
\begin{split}
T\left[
\hat{q}^{H}_{\underline{k}}(t_1)
\hat{q}^{H}_{\underline{k}^{\prime}}(t_2)
\right]
=&
\thinspace
\hat{q}^{H}_{\underline{k}}(t_1)
\hat{q}^{H}_{\underline{k}^{\prime}}(t_2)
\theta(t_{1}-t_{2})
\\
\thinspace\thinspace
+&
\thinspace
\hat{q}^{H}_{\underline{k}^{\prime}}(t_2)
\hat{q}^{H}_{\underline{k}}(t_1)
\theta(t_{2}-t_{1})
.
\end{split}
\end{equation}
In this form the following steps can be visualized and extended to the product of more operators, much easier. At first we apply Eq.\enskip\eqref{1st_expression_for_Omega_H} to the later time in each term:
\begin{equation}
\begin{split}
T\left[
\hat{q}^{H}_{\underline{k}}(t_1)
\hat{q}^{H}_{\underline{k}^{\prime}}(t_2)
\right]
=
\hat{S}^{-1}
\Big\{
T\left(  e^{-i\int_{t_{1}}^{\infty}\mathrm{d}t^{\prime}_{1} \hat{V}_{in}(t_{1}^{\prime})}\right)\hat{q}^{in}_{\underline{k}}(t_1)\hat{\Omega}_{in}(t_1) \hat{q}^{H}_{\underline{k}^{\prime}}(t_2)   \theta(t_1 - t_2)
		\\
    + T\left(  e^{-i\int_{t_{2}}^{\infty}\mathrm{d}t^{\prime}_{2} \hat{V}_{in}(t_{2}^{\prime})}\right)\hat{q}^{in}_{\underline{k}^{\prime}}(t_2)\hat{\Omega}_{in}(t_2) 
      \hat{q}^{H}_{\underline{k}}(t_1)   \theta(t_2 - t_1)
    \Big\}
    .  
    \end{split} 
\end{equation}
One can immediately see, that one $ \hat{S}^{-1} $-operator can be moved out from both lines. Furthermore, the conditions of the cases ensures that the respective operators to the right can be moved inside of the inner chronological time ordering. This is similar to what was done between Eq.\enskip\eqref{Moving_f_inside} and \eqref{3rd_expression_for_Omega_H}. 
\begin{equation}
\begin{split}
T\left[
\hat{q}^{H}_{\underline{k}}(t_1)
\hat{q}^{H}_{\underline{k}^{\prime}}(t_2)
\right]
=
\hat{S}^{-1}
&\Big\{
T\left(\hat{q}^{in}_{\underline{k}}(t_1)  e^{-i\int_{t_{1}}^{\infty}\mathrm{d}t^{\prime}_{1} \hat{V}_{in}(t_{1}^{\prime})}\hat{\Omega}_{in}(t_1) \hat{q}^{H}_{\underline{k}^{\prime}}(t_2)\right)   \theta(t_1 - t_2)
		\\
    &+T\Big( \hat{q}^{in}_{\underline{k}^{\prime}}(t_2)  
    \underbrace{
    e^{-i\int_{t_{2}}^{\infty}\mathrm{d}t^{\prime}_{2} \hat{V}_{in}(t_{2}^{\prime})}\hat{\Omega}_{in}(t_2)  }_{=\hat{S}}
    \hat{q}^{H}_{\underline{k}}(t_1)
    \Big)
        \theta(t_2 - t_1)
    \Big\}.  
    \end{split}
\end{equation}
Observe, that $ \hat{S} $ can be derived from the product of the exponential function and the unitary operator $ \hat{\Omega}_{in} $(see Eq.\enskip\eqref{Move_in_T_from_right} and below). The $ \hat{Q}_{H}$ operator must be expressed using Eq.\enskip\eqref{1st_expression_for_Omega_H}:
\begin{equation}
\begin{split}
&=\hat{S}^{-1}
\Big\{
T\left[
	\hat{q}^{in}_{\underline{k}}(t_1)\hat{S} \hat{S}^{-1}
	\hat{q}^{in}_{\underline{k}^{\prime}}(t_2)
 	T\left( 
 		e^{-i\int_{t_{2}}^{\infty}\mathrm{d}t_{2}^{\prime} \hat{V}_{in}(t_{2}^{\prime})}
	\right)
 	T\left( 
 		e^{-i\int_{-\infty}^{t_{2}}\mathrm{d}t_{2}^{\prime} \hat{V}_{in}(t_{2}^{\prime})}
 	\right)
\right]
				  \theta(t_1 - t_2)
		\\
&\qquad\qquad +T\Big[
	\hat{q}^{in}_{\underline{k}^{\prime}}(t_2)\underbrace{\hat{S} \hat{S}^{-1}}_{=1}
	\hat{q}^{in}_{\underline{k}}(t_1)
 	T\left( 
 		e^{-i\int_{t_{1}}^{\infty}\mathrm{d}t_{1}^{\prime} \hat{V}_{in}(t_{1}^{\prime})}
	\right)
 	T\left( 
 		e^{-i\int_{-\infty}^{t_{1}}\mathrm{d}t_{1}^{\prime} \hat{V}_{in}(t_{1}^{\prime})}
 	\right)
\Big]
        \theta(t_2 - t_1).
    \Big\}  
    \end{split}
\end{equation}
After cancelling the scattering operator and its inverse, we now merging both exponential function regarding the earlier time:
\begin{equation}
\begin{split}
T\left[
\hat{q}^{H}_{\underline{k}}(t_1)
\hat{q}^{H}_{\underline{k}^{\prime}}(t_2)
\right]=
&\hat{S}^{-1}
\Big\{
T\left[
	\hat{q}^{in}_{\underline{k}}(t_1)
	\hat{q}^{in}_{\underline{k}^{\prime}}(t_2)
	\hat{S}
\right]
				  \theta(t_1 - t_2)
	\\				  
				 &\quad +
T\left[
	\hat{q}^{in}_{\underline{k}^{\prime}}(t_2)
	\hat{q}^{in}_{\underline{k}}(t_1)
	\hat{S}
\right]
        \theta(t_2 - t_1).
    \Big\}  
    \end{split}
\end{equation}
Note, rewriting our notation using cases back to the chronological time operator $ T $, results in $ T(T[\mydots]) $. After dropping the redundant second time ordering operator, it reads:
\begin{equation}\label{Gell-Mann_Low_formula_for_2}
\begin{split}
T\left( 
\hat{q}^{H}_{\underline{k}}(t_1)
\hat{q}^{H}_{\underline{k}^{\prime}}(t_2)
\right) 
=
\hat{S}^{-1}
T\left( 
\hat{q}^{in}_{\underline{k}}(t_1)
\hat{q}^{in}_{\underline{k}^{\prime}}(t_2)
\hat{S}
\right).
    \end{split}
\end{equation}
Following these step multiple time, one can easily  arrive at the final form of the Gell-Mann Low formula:
\begin{equation}\label{Gell-Mann_Low_formula}
\begin{split}
\textcolor[RGB]{0,0,204}{
T\left( 
\hat{q}^{H}_{\underline{k}}(t_1)
\hat{q}^{H}_{\underline{k}^{\prime}}(t_2)
\ldots
\right) 
=
\hat{S}^{-1}
T\left( 
\hat{q}^{in}_{\underline{k}}(t_1)
\hat{q}^{in}_{\underline{k}^{\prime}}(t_2)
\ldots
\hat{S}
\right).
}
    \end{split}
\end{equation}
Its major importance in quantum field theory comes, when applying perturbation theory. When applied between asymptotic vacuum states of the $ in $ and $ out $ picture, it reads:
\begin{equation}\label{GML_applied_0.1}
\begin{split}
&\bra{0_{out}}
T\left( 
\hat{q}^{H}_{\underline{k}}(t_1)
\hat{q}^{H}_{\underline{k}^{\prime}}(t_2)
\ldots
\right) 
\ket{0_{in}}
\\
&=
\bra{0_{out}}
\hat{S}^{-1}
T\left( 
\hat{q}^{in}_{\underline{k}}(t_1)
\hat{q}^{in}_{\underline{k}^{\prime}}(t_2)
\ldots
\hat{S}
\right)
\ket{0_{in}}
.
\end{split}
\end{equation}
The inverse of the scattering operator allows to transform these vacuum states like Eq.\enskip\eqref{in_to_out_state}: $\bra{0_{out}}\hat{S}^{-1}=\bra{0_{in}}$. One important aspect to consider at this point are disconnected graphs. These are terms containing vacuum fluctuations not linked to incoming and out-going particles [vacuum bubbles]. In order to avoid these highly divergent objects, it is convenient to work out the connected Green's functions. For this, we have to divide Eq.\enskip\eqref{GML_applied_0.1} by $ \braket{0_{out}|0_{in}} = \braket{0_{in}|\hat{S}|0_{in}} $ (see $ [1]$).
\newpage
\chapter{Conclusion}
In this report we have presented a set of fundamental concepts and methods. These are essential for starting to study Quantum field theory. We began with well-known pictures from quantum mechanics and introduced the $ in $ and $ out $ picture as consequences of working with asymptotic states found by investigating the effect of an external current on an scalar action. Establishing connections between the pictures on the level of interaction states was based on the scattering operator $ \hat{S} $. With this object, we could derive the Gell-Mann Low formula, which plays a major part in the second quantisation and perturbation theory performed in context of quantum field theory. 
%%
%%
\newpage
\begin{subappendices}
\section{Chronological and Anti-Chronological time ordering}\label{chronological_time}

To expand our concept  of chronological time ordering to cases of more then  two $  t $ we start by advancing the definition stated in Eq.\enskip\eqref{eq:chron-time_ordering}.
In general time ordering consists of a summation of all permutations $ P $ of a given set multiplied by Heaviside functions. The notation $ P_{j} $ refers to a specific arrangement, while $ P_{j}(i) $ returns the index of the operator at this position $ i $ in the permutation $ j $. 
We keep the number of brackets low by denoting $ \hat{V}_{I}(t_{1}) \rightarrow \hat{V}_{1} $ \\
Definition:
\begin{equation}\label{T_general}
\textcolor[RGB]{0,0,204}{
T(\hat{V}_1 \hat{V}_2\ldots\hat{V}_n)=
\sum_{j=1}^{n!}\ P_{j}
\left[
\prod^{n}_{l}\hat{V}_{l}
 \right]  
\left[
\prod^{n-1}_{i}
\theta
	\left(
	t_{P_{j}(i)}-t_{P_{j}(i+1) }
	\right)
\right]
.
}
\end{equation}
The generalisation of anti-chronological time ordering given in Eq.\enskip\eqref{eq:anti-chron-time_ordering} is based on the same principle. 
Definition:
\begin{equation}\label{T_bar_general}
\textcolor[RGB]{0,0,204}{
\bar{T}(\hat{V}_1, \hat{V}_2,\ldots,\hat{V}_n)=
\sum_{j=1}^{n!}\ P_{j}
\left[
\prod^{n}_{l}\hat{V}_{l}
 \right]  
\left[
\prod^{n-1}_{i}
\theta
	\left(
	t_{P_{j}(i+1)}-t_{P_{j}(i)}
	\right)
\right]
 .
}
\end{equation}
%But this summation over $ j $ would not add up to the same value for $ I(t) $ since it started in the order of one element in the sum.In this case we need a normalization in addition to $ T $ . From statistics we know a set of n different elements can be linear arranged in $ n! $ ways.Coming in as a factor of $ \frac{1}{n!} $ in the expressions later.
As a test, we set $ n=3 $ in Eq.\enskip\eqref{T_general}:
\begin{equation}
\begin{split}
T(\hat{V}_1, \hat{V}_2,\hat{V}_3)
&=
\sum_{j=1}^{3!}\ P_{j}
\left[
\hat{V}_1 \hat{V}_2\hat{V}_3
 \right]  
\left[
\prod^{3-1}_{i}
\theta
	\left(
	t_{P_{j}(i)}-t_{P_{j}(i+1) }
	\right)
\right]
\\
&=\enspace\thinspace\thinspace
\hat{V}_{1}\hat{V}_{2}\hat{V}_{3}\  \theta(t_{1}-t_{2}) \theta(t_{2}-t_{3})
	\\
	& \ \ \ +\hat{V}_{1}\hat{V}_{3}\hat{V}_{2}\   \theta(t_{1}-t_{3})\theta(t_{3}-t_{2})
	\\
	& \ \ \ +\hat{V}_{2}\hat{V}_{1}\hat{V}_{3} \  \theta(t_{2}-t_{1})\theta(t_{1}-t_{3})
	\\
	& \ \ \ +\hat{V}_{2}\hat{V}_{3}\hat{V}_{1}\   \theta(t_{2}-t_{3})\theta(t_{3}-t_{1})
	\\
	& \ \ \ +\hat{V}_{3}\hat{V}_{1}\hat{V}_{2}\   \theta(t_{3}-t_{1})\theta(t_{1}-t_{2})
	\\
	& \ \ \ +\hat{V}_{3}\hat{V}_{2}\hat{V}_{1}\  \theta(t_{3}-t_{2})\theta(t_{2}-t_{1}).
\end{split}
\end{equation}
As well as $ n=3 $ in Eq.\enskip\eqref{T_bar_general}
\begin{equation}
\begin{split}
\bar{T}(\hat{V}_1, \hat{V}_2,\hat{V}_3)
&=
\sum_{j=1}^{3!}\ P_{j}
\left[
\hat{V}_1, \hat{V}_2,\hat{V}_3
 \right]  
\left[
\prod^{3-1}_{i}
\theta
	\left(
	t_{P_{j}(i+1)}-t_{P_{j}(i)}
	\right)
\right]\
\\
&=\enspace\thinspace\thinspace
\hat{V}_{1}\hat{V}_{2}\hat{V}_{3}\  \theta(t_2 -t_1)\theta(t_3 -t_2)
	\\
	& \ \ \ +\hat{V}_{1}\hat{V}_{3}\hat{V}_{2}\   \theta(t_3 -t_1)\theta(t_2 -t_3)
	\\
	& \ \ \ +\hat{V}_{2}\hat{V}_{1}\hat{V}_{3}\  \theta(t_1 -t_2)\theta(t_3 -t_1)
	\\
	& \ \ \ +\hat{V}_{2}\hat{V}_{3}\hat{V}_{1}\   \theta(t_3 -t_2)\theta(t_1 -t_3)
	\\
	& \ \ \ +\hat{V}_{3}\hat{V}_{1}\hat{V}_{2} \  \theta(t_1 -t_3)\theta(t_2 -t_1)
	\\
	& \ \ \ +\hat{V}_{3}\hat{V}_{2}\hat{V}_{1} \  \theta(t_2 -t_3)\theta(t_1 -t_2)
.
\end{split}
\end{equation}
The following results and calculations can be performed in the same fashion for chronological and anti-chronological time ordering. We will show them explicitly for the case of chronological time ordering.

In Eq.\enskip\eqref{regular_function} we defined an function $ I(t)$ for two potentials $ V_I(t) $. Expanding this function to the case of $ n $ requires a normalization factor alongside $ T $. A set of n different elements can be linear arranged in $ n! $ ways. Therefore a factor of $ 1/{n!} $ in the expressions for $ I(t)$ is required, when chronological time ordering is applied.  

Now we use Eq.\enskip\eqref{T_general} in a larger $ I(t) $ and perform a proof by induction based on the number of $ \hat{V} $. The induction start is the case of two $ \hat{V} $. In the induction step we state that it works for at least one unspecified higher order $ k $: ($ \textit{Note:} $ $ t_0 = t $)\\
\begin{equation}
I_k (t)
=
 \prod_{a=1}^{k} 
 \int_{-\infty}^{t_{a-1}}\mathrm{d}t_a\
  \hat{V}_a
=
\dfrac{1}{k!}
 (
 \prod_{a=1}^{k} 
\int_{-\infty}^{t}\mathrm{d}t_a\
)
T(\hat{V}_1,\ldots,\hat{V}_k).
\end{equation}
Moving one increment higher in our 'chain' $ k+1 $:
\begin{equation}
I_{k+1} (t)
=
 \prod_{a=1}^{k+1} 
 \int_{-\infty}^{t_{a-1}}\mathrm{d}t_a\
  \hat{V}_a
  =
   \prod_{a=1}^{k} 
 \int_{-\infty}^{t_{a-1}}\mathrm{d}t_a\
  \hat{V}_a
  \cdot
 \int_{-\infty}^{t_{k}}\mathrm{d}t_{k+1}\
 \hat{V}_{k+1}.
\end{equation}
Using the induction step and general definition Eq.\enskip\eqref{T_general}:
\begin{equation}
I_{k+1} (t)
=
\dfrac{1}{k!}
 (
 \prod_{a=1}^{k} 
\int_{-\infty}^{t}\mathrm{d}t_a\
)
T(\hat{V}_1,\ldots,\hat{V}_k)
\cdot
 \int_{-\infty}^{t_{k}}\mathrm{d}t_{k+1}\
 \hat{V}_{k+1}
\end{equation}
\begin{equation}
I_{k+1} (t)
=
I_k (t)
\cdot
 \int_{-\infty}^{t_{k}}\mathrm{d}t_{k+1}\
 \hat{V}_{k+1}
\end{equation}
This shows that the incrementation of $ k $ reduces to a multiplication with one more different element to the set. This increases the possible permutations by a factor of $ k+1 $ resulting in $ (k+1)! $ in total.
\begin{equation}
I_{k+1}(t)
=
\dfrac{1}{(k+1)!}
 (
 \prod_{a=1}^{k+1} 
\int_{-\infty}^{t}\mathrm{d}t_a\
)
T(\hat{V}_1,\ldots,\hat{V}_{k+1})
\end{equation}
%Time ordering is as concept in field theory t. For expressing the propagators in term of vaccum expectation values of fields requires. % The reason apart from pure mathematics is comprehensible. The different functions, functionals or fields are best organized for summarizing a scattering or interaction event if they are time-like sorted.

An very useful aspect of chronological as well as anti-chronological time ordering is all operators $ V_i $ in $ T(\ldots) $ or $ \bar{T}(\ldots) $ commute. For two elements:
\begin{subequations}
\begin{align}
T(\hat{V}_1, \hat{V}_2)=\hat{V}_1\ \hat{V}_2\ \theta (t_1 -t_2)\ +\ \hat{V}_2\  \hat{V}_1 \ \theta (t_2-t_1)
\\
T(\hat{V}_2, \hat{V}_1)=\hat{V}_2\ \hat{V}_1\ \theta (t_2 -t_1)\ +\ \hat{V}_1\  \hat{V}_2 \ \theta (t_1-t_2)
,
\end{align}
\end{subequations}
after switching the terms,
\begin{equation}\textcolor[RGB]{0,0,204}{
T(\hat{V}_1, \hat{V}_2)=T(\hat{V}_2, \hat{V}_1)
}
\end{equation}
In other words, the commutation relations say whether the subtraction of permutations of elements is zero or not. But in time ordering all permutations appear, we can rearrange the terms so subtraction of equal permutations happens. Therefore commutation holds for more then two $ V_i $.
\end{subappendices}
\newpage
\begin{thebibliography}{12}
 \bibitem{aheaehw} 
S.Randjbar-Daemi.
\textit{Course on Quantum Electrodynamics: Introduction to Quantum Field Theory}.
 The Abdus Salam International Centre for Theoretical Physics , 2007-2008.
 
\bibitem{latexcompanion} 
M.Peskin; D.Schroeder. 
\textit{Quantum field theory}. 
Perseus Books Publishing, 1995.
 
\bibitem{einstein} 
W.Greiner;J.Reinhart.
\textit{Quantum electrodynamics}.
 Verlag Harri Deutsch Thun und Frankfurt am Main, 1984.
 
 \bibitem{gw} 
W.Greiner;J.Reinhart.
\textit{Feldquantisierung}.
 Verlag Harri Deutsch Thun und Frankfurt am Main, 1993.
 
 \bibitem{iz} 
C.Itzykson;J.Zuber.
\textit{Quantum Field Theory}.
 McGraw-Hill Book Company, International Edition 1985. 
 
\end{thebibliography}

%\appendix % Cue to tell LaTeX that the following "chapters" are Appendices

% Include the appendices of the thesis as separate files from the Appendices folder
% Uncomment the lines as you write the Appendices

%\include{Appendices/AppendixA}
%\include{Appendices/AppendixB}
%\include{Appendices/AppendixC}

%----------------------------------------------------------------------------------------
%	BIBLIOGRAPHY
%----------------------------------------------------------------------------------------

%\printbibliography[heading=bibintoc]

%----------------------------------------------------------------------------------------

\end{document}  
